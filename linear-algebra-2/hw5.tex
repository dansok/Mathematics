\documentclass[11pt]{article}
\usepackage{amsmath,amssymb,amsthm}
\addtolength{\evensidemargin}{-.5in}
\addtolength{\oddsidemargin}{-.5in}
\addtolength{\textwidth}{0.8in}
\addtolength{\textheight}{0.8in}
\addtolength{\topmargin}{-.4in}
\newtheoremstyle{quest}{\topsep}{\topsep}{}{}{\bfseries}{}{ }{\thmname{#1}\thmnote{ #3}.}
\theoremstyle{quest}
\newtheorem*{definition}{Definition}
\newtheorem*{theorem}{Theorem}
\newtheorem*{question}{Question}
\newtheorem*{exercise}{Exercise}
\newtheorem*{challengeproblem}{Challenge Problem}
\newcommand{\name}{%%%%%%%%%%%%%%%%%%
%%%%%%%%%%%%%%%%%%%%%%%%%%%%%%
%%%%%%%%%%%%%%%%%%%%%%%%%%%%%%
%% put your name here, so we know who to give credit to %%
Dan Sokolsky
}%%%%%%%%%%%%%%%%%%%%%%%%%%%%%%
\newcommand{\hw}{%%%%%%%%%%%%%%%%%%%%
%% and which homework assignment is it? %%%%%%%%%
%% put the correct number below              %%%%%%%%%
%%%%%%%%%%%%%%%%%%%%%%%%%%%%%%
5
}
%%%%%%%%%%%%%%%%%%%%%%%%%%%%%%
%%%%%%%%%%%%%%%%%%%%%%%%%%%%%%
%%%%%%%%%%%%%%%%%%%%%%%%%%%%%%
\title{\vspace{-50pt}
\Huge \name
\\\vspace{20pt}
\huge Linear Algebra II\hfill Homework \hw}
\author{}
\date{}
\pagestyle{myheadings}
\markright{\name\hfill Homework \hw\qquad\hfill}

%% If you want to define a new command, you can do it like this:
\newcommand{\Q}{\mathbb{Q}}
\newcommand{\R}{\mathbb{R}}
\newcommand{\Z}{\mathbb{Z}}
\newcommand{\C}{\mathbb{C}}

%% If you want to use a function like ''sin'' or ''cos'', you can do it like this
%% (we probably won't have much use for this)
% \DeclareMathOperator{\sin}{sin}   %% just an example (it's already defined)


\begin{document}
\maketitle

\begin{question}[1]
Let $X$ be a linear space such that $\oplus_j V_j = X = \oplus_j U_j$. Prove $U_j \subset V_j \implies U_j = V_j$.
\end{question}
\begin{proof}
$V_i \cap V_j = \{0\}$ for $i \ne j$. In particular, $U_j \subset V_j \implies U_j \cap V_i = \{0\}$ for $i \ne j$. Thus $x \in V_j \setminus U_j \implies x \notin U_i$ for any $i$. Thus $x \notin \oplus_j U_j = X = \oplus_j V_j$. That is, $x \notin V_i$ for any $i$, a contradiction. Therefore, $V_j \setminus U_j = \emptyset$. Thus, $U_j = V_j$. 
\end{proof}
\begin{question}[4]
Let $A$ be $n \times n$ of rank 2. Classify the minimal polynomial of $A$.
\end{question}
\begin{proof}
$n \ge 2$ since $rank(A) = 2$.
\\$\underline{If\ n = 2}$, then either $A$ is diagonalizable, or it is similar to a $2 \times 2$ Jordan block. If it is diagonalizable, then -- if the eigenvalues are distinct, $m_A(t) = (t - \lambda_1)(t - \lambda_2)$. If the eigenvalue is repeated, $m_A(t) = t - \lambda_1$ (there are two $1 \times 1$ Jordan blocks for the eigenvalue $\lambda_1$). Else, If it is similar to a Jordan block, $m_A(t) = (t - \lambda_1)^2$. In either case, $\lambda_1, \lambda_2 \ne 0$ since if even a single eigenvalue is $0$, say $\lambda_1 = 0$, then it follows that $rank(A) \le 1$, contradicting $rank(A)=2$.
\\$\underline{If\ n \ge 3}$, then $\Big A \sim \begin{bmatrix} X & 0 \\ 0 & 0 \end{bmatrix}$ (the Jordan Normal Form) under the basis $\{v_1, v_2, w_1,\ldots,w_{n-2}\}$; where $\{v_1, v_2\}$ are basis elements for the (generalized) eigenspace(s) that make up $X$ and $\{w_i\}_{i \le n-2}$ are linearly independent, and don't belong in  the eigenspace(s) that make up $X$. Thus $X$ is $2 \times 2$, $rank(X)=2$, and the nature of it's minimal polynomial has been discussed in the $n=2$ case. The lower right block matrix is a diagonal matrix with all $0$s on the diagonal ($1 \times 1$ Jordan blocks). Thus the minimal polynomial may have one of the following forms -- $m_A(t) = t(t - \lambda_1)(t - \lambda_2)$; $m_A(t) = t(t - \lambda_1)$; or $m_A(t) = t(t - \lambda_1)^2$.
\end{proof}

\end{document}
