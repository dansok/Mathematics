\documentclass[11pt]{article}
\usepackage{amsmath,amssymb,amsthm}
\addtolength{\evensidemargin}{-.5in}
\addtolength{\oddsidemargin}{-.5in}
\addtolength{\textwidth}{0.8in}
\addtolength{\textheight}{0.8in}
\addtolength{\topmargin}{-.4in}
\newtheoremstyle{quest}{\topsep}{\topsep}{}{}{\bfseries}{}{ }{\thmname{#1}\thmnote{ #3}.}
\theoremstyle{quest}
\newtheorem*{definition}{Definition}
\newtheorem*{theorem}{Theorem}
\newtheorem*{question}{Question}
\newtheorem*{exercise}{Exercise}
\newtheorem*{challengeproblem}{Challenge Problem}
\newcommand{\name}{%%%%%%%%%%%%%%%%%%
%%%%%%%%%%%%%%%%%%%%%%%%%%%%%%
%%%%%%%%%%%%%%%%%%%%%%%%%%%%%%
%% put your name here, so we know who to give credit to %%
Dan Sokolsky
}%%%%%%%%%%%%%%%%%%%%%%%%%%%%%%
\newcommand{\hw}{%%%%%%%%%%%%%%%%%%%%
%% and which homework assignment is it? %%%%%%%%%
%% put the correct number below              %%%%%%%%%
%%%%%%%%%%%%%%%%%%%%%%%%%%%%%%
2
}
%%%%%%%%%%%%%%%%%%%%%%%%%%%%%%
%%%%%%%%%%%%%%%%%%%%%%%%%%%%%%
%%%%%%%%%%%%%%%%%%%%%%%%%%%%%%
\title{\vspace{-50pt}
\Huge \name
\\\vspace{20pt}
\huge Linear Algebra II\hfill Homework \hw}
\author{}
\date{}
\pagestyle{myheadings}
\markright{\name\hfill Homework \hw\qquad\hfill}

%% If you want to define a new command, you can do it like this:
\newcommand{\Q}{\mathbb{Q}}
\newcommand{\R}{\mathbb{R}}
\newcommand{\Z}{\mathbb{Z}}
\newcommand{\C}{\mathbb{C}}

%% If you want to use a function like ''sin'' or ''cos'', you can do it like this
%% (we probably won't have much use for this)
% \DeclareMathOperator{\sin}{sin}   %% just an example (it's already defined)


\begin{document}
\maketitle

\begin{question}[1]
Prove $\cos(\alpha -\beta) = \cos(\alpha)\cos(\beta)+\sin(\alpha)\sin(\beta)$
\end{question}
\begin{proof}
Let $x, y$ be points on the unit circle with coordinates $x = (\cos\beta, sin\beta),\\ y = (\cos\alpha, \sin\alpha)$. Then $||x|| = ||y|| = 1$, and $\angle(x, y) = \alpha-\beta$. Thus, by formula (5),
$$\cos(\alpha)\cos(\beta)+\sin(\alpha)\sin(\beta) = \langle x, y\rangle = ||x||\ ||y||\cos(\alpha-\beta) = 1 \cdot 1 \cdot \cos(\alpha-\beta) = \cos(\alpha-\beta)$$
\end{proof}
\begin{question}[2]
Prove $u \times v = ||u||\ ||v|| \sin\theta$; where $\theta := \angle(u, v)$.
\end{question}
\begin{proof}
$\langle u \times v, w \rangle = \det (u, v, w) = \det \begin{bmatrix} u_1 & v_1 & w_1\\ u_2 & v_2 & w_2 \\ u_3 & v_3 & w_3 \end{bmatrix} = w_1(u_2v_3 - u_3v_2) - w_2(u_1v_3 - u_3v_1) + w_3(u_1v_2 - u_2v_1) = w_1(u_2v_3 - u_3v_2) + w_2(u_3v_1 - u_1v_3) + w_3(u_1v_2 - u_2v_1) = \Big\langle\begin{bmatrix}u_2v_3 - u_3v_2 \\ u_3v_1 - u_1v_3 \\ u_1v_2 - u_2v_1
\end{bmatrix}, w \Big\rangle \iff u \times v = \begin{bmatrix}u_2v_3 - u_3v_2 \\ u_3v_1 - u_1v_3 \\ u_1v_2 - u_2v_1
\end{bmatrix}$. Thus, $$||u \times v||^2 = \langle u \times v, u \times v \rangle = (u_2v_3 - u_3v_2)^2 + (u_3v_1 - u_1v_3)^2 + (u_1v_2 - u_2v_1)^2 =$$ $$u_2^2 v_1^2 + u_3^2 v_1^2 - 2 u_1 u_2 v_2 v_1 - 2 u_1 u_3 v_3 v_1 + u_1^2 v_2^2 + u_3^2 v_2^2 + u_1^2 v_3^2 + u_2^2 v_3^2 - 2 u_2 u_3 v_2 v_3\ \ (1)$$
Additionally,
$$(||u||\ ||v||)^2 =(u_1^2 + u_2^2 + u_3^2)(v_1^2 + v_2^2 + v_3^2) = $$ $$u_1^2 v_1^2 + u_2^2 v_1^2 + u_3^2 v_1^2 + u_1^2 v_2^2 + u_2^2 v_2^2 + u_3^2 v_2^2 + u_1^2 v_3^2 + u_2^2 v_3^2 + u_3^2 v_3^2 \ \ (2)$$
Also,
$$\langle u, v \rangle ^2 = (u_1v_1 + u_2v_2 + u_3v_3)^2 = u_1^2 v_1^2 + 2 u_1 u_2 v_2 v_1 + 2 u_1 u_3 v_3 v_1 + u_2^2 v_2^2 + u_3^2 v_3^2 + 2 u_2 u_3 v_2 v_3\ \ (3)$$
and notice that (1) = (2) - (3). That is,
$$||u \times v||^2 = (||u||\ ||v||)^2 - \langle u, v \rangle ^2 = (||u||\ ||v||)^2 - (||u||\ ||v|| \cos\theta)^2 =$$ $$||u||^2||v||^2 (1 - (\cos\theta)^2) = ||u||^2||v||^2 (\sin \theta)^2$$
$\iff ||u \times v|| = ||u||\ ||v|| \sin \theta$
\end{proof}
\begin{question}[3]
Can the standard basis $e_1 = (1, 0, \ldots, 0), e_2 = (0, 1, \ldots, 0), \ldots, e_n = (0, 0, \ldots, 1)$ be used as a basis for $\mathbb{C}^n$?
\end{question}
\begin{proof}
The answer depends on the particular field, $\mathbb{F}$, the  vector space $\mathbb{C}^n$ is defined over. If $\mathbb{F} = \mathbb{C}$, then $\{e_1, \ldots, e_n\}$ does in fact form a basis for $\mathbb{C}^n$. If $\mathbb{F} \subsetneq \mathbb{C}$, say $\mathbb{F} = \mathbb{R}$, then $\{e_1, \ldots, e_n\}$ doesn't form a basis for $\mathbb{C}^n$, since it doesn't even span $\mathbb{C}^n$. E.g., $i \notin span\{e_1, \ldots, e_n\}$.
\end{proof}
\begin{question}[4]
Let $K$ be an $n \times n$ upper-triangular matrix, with all zeros on the diagonal. Let $J$ be an $n \times n$ Jordan block, with ones on the superdiagonal. Let $A = [a_1, \ldots, a_n], \\a_j \in \mathbb{C}^n$ be an $n \times n$ matrix.
\\(a) Show there exists $m \le n$ such that $K^m = 0$.
\\(b) Show $AJ = [0, a_1, \ldots, a_{n-1}]$.
\\(c) Show $J \sim J^T$.
\end{question}
\begin{proof} $\\ $
(a) Each $0$-column annihilates the next column. Since there are $n$ columns, $m \le n$.
\\(b) The first column of $AJ$ are all zeros, due to the right-action of the first $0-$column of $J$. For the rest of the columns, i.e., $j \ge 1$, observe --
$$(AJ)_{i,j} = \sum_{k=1}^n A_{i,k}J_{k,j} = A_{i,j-1}\cdot J_{j-1,j} = A_{i,j-1} \cdot 1 = A_{i,j-1}$$
That is, $AJ = [0, a_1, \ldots, a_{n-1}]$.
\\(c)Define the $P$ to be an $n \times n$ matrix defined by $P_{i,n-i+1} = 1$ and $P_{i,j} = 0$ for $j \ne n-i+1$. Then $(P^2)_{i,j} = \sum_{k=1}^n P_{i,k}P_{k,j} = P_{i,n-i+1}\cdot P_{n-j+1, j} = 1 \cdot 1 = 1 \iff n-i+1 = n-j+1 \iff i=j$; and $(P^2)_{i,j} = 0$ if $i \ne j$. I.e., $P^2 = I_n$. That is, $P = P^{-1}$. Now,
a left-action by $P$ reverses the order of the rows of $J$, and a right-action by $P$ reverses the order of the columns of $J$. Together, $PJP^{-1} = PJP = J^T$. That is, $J \sim J^T$.
\end{proof}
\begin{question}[5]
Let $A: X \rightarrow X$. Let $Y \subset X$ be $A-$invariant. Show $Y \cap N_A = \{0\} \implies A \restriction_Y: Y \rightarrow Y$ is invertible.
\end{question}
\begin{proof}
Suppose $Y \cap N_A = \{0\}$. Then $0 \ne y \in Y \implies 0 \ne Ay \in Y$. That is, $N_{A \restriction_Y} = \{0\}$. By the Rank-Nullity theorem, it follows that $A \restriction_Y$ has full rank $\iff A \restriction_Y$ is invertible.
\end{proof}

\end{document}
