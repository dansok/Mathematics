\documentclass[11pt]{article}
\usepackage{amsmath,amssymb,amsthm, mathrsfs}
\addtolength{\evensidemargin}{-.5in}
\addtolength{\oddsidemargin}{-.5in}
\addtolength{\textwidth}{0.8in}
\addtolength{\textheight}{0.8in}
\addtolength{\topmargin}{-.4in}
\newtheoremstyle{quest}{\topsep}{\topsep}{}{}{\bfseries}{}{ }{\thmname{#1}\thmnote{ #3}.}
\theoremstyle{quest}
\newtheorem*{definition}{Definition}
\newtheorem*{theorem}{Theorem}
\newtheorem*{question}{Question}
\newtheorem*{exercise}{Exercise}
\newtheorem*{challengeproblem}{Challenge Problem}
\newcommand{\name}{%%%%%%%%%%%%%%%%%%
%%%%%%%%%%%%%%%%%%%%%%%%%%%%%%
%%%%%%%%%%%%%%%%%%%%%%%%%%%%%%
%% put your name here, so we know who to give credit to %%
Dan Sokolsky
}%%%%%%%%%%%%%%%%%%%%%%%%%%%%%%
\newcommand{\hw}{%%%%%%%%%%%%%%%%%%%%
%% and which homework assignment is it? %%%%%%%%%
%% put the correct number below              %%%%%%%%%
%%%%%%%%%%%%%%%%%%%%%%%%%%%%%%
8
}
%%%%%%%%%%%%%%%%%%%%%%%%%%%%%%
%%%%%%%%%%%%%%%%%%%%%%%%%%%%%%
%%%%%%%%%%%%%%%%%%%%%%%%%%%%%%
\title{\vspace{-50pt}
\Huge \name
\\\vspace{20pt}
\huge Linear Algebra II\hfill Homework \hw}
\author{}
\date{}
\pagestyle{myheadings}
\markright{\name\hfill Homework \hw\qquad\hfill}

%% If you want to define a new command, you can do it like this:
\newcommand{\Q}{\mathbb{Q}}
\newcommand{\R}{\mathbb{R}}
\newcommand{\Z}{\mathbb{Z}}
\newcommand{\C}{\mathbb{C}}

%% If you want to use a function like ''sin'' or ''cos'', you can do it like this
%% (we probably won't have much use for this)
% \DeclareMathOperator{\sin}{sin}   %% just an example (it's already defined)


\begin{document}
\maketitle

\begin{question}[1]
(a) Let $A, B \in \mathbb{R}^{m \times n}$. Prove $B^TA = vec(A) \cdot vec(B)$.
\\(b) Prove $||A + B||_F \le ||A||_F + ||B||_F$
\end{question}
\begin{proof}
$\mathbf{(a)}$ Observe that the $i^{th}$ row of $B^T$ is the $i^{th}$ column of $B$. Further observe $B^TA$ is $n \times n$. Denote by $a_j,\ b_j$ the $j^{th}$ columns of $A, B$, respectively.
\\Thus, $(B^TA)_{i,j} = \langle a_i, b_j \rangle$. Therefore,
$$tr(B^TA) = \sum_{j=1}^n \langle a_j, b_j \rangle = \sum_{i=1}^m \sum_{j=1}^n A_{i, j} \cdot B_{i, j} = \langle vec(A), vec(B)\rangle$$
\\$\mathbf{(b)}$
$$||A+B||_F = ||vec(A+B)||_2 = ||vec(A) + vec(B)||_2$$ $$\le ||vec(A)||_2 + ||vec(B)||_2 = ||A||_F + ||B||_F$$
\end{proof}
\begin{question}[2]
(a) Prove $||A^TA||_F \le ||A||_F^2$
\\(b) Let $A$ be $m \times k$. Let $B$ be $k \times n$. Prove $||AB||_F \le ||A||_F||B||_F$
\end{question}
\begin{proof}
$\mathbf{(a)}$ Let $a_j$ denote the $j^{th}$ column of $A$. As before, the $i^{th}$ row of $A^T$ is the $i^{th}$ column of $A$. Thus, $(A^TA)_{i,j} = \langle a_i, a_j \rangle$. Thus,
$$||A^TA||_F = \sqrt{\sum_{1 \le i,j \le n} \langle a_i, a_j \rangle^2} \le \sqrt{\sum_{1 \le i,j \le n} ||a_i||^2 \cdot ||a_j||^2} = $$ $$\sqrt{\sum_{i=1}^n \sum_{j=1}^n ||a_i||^2 \cdot ||a_j||^2} = \sqrt{\sum_{i=1}^n ||a_i||^2 \cdot \sum_{j=1}^n  ||a_j||^2} = ||A||_F \cdot ||A||_F = ||A||_F^2$$
\\$\mathbf{(b)}$
\end{proof}

\end{document}
