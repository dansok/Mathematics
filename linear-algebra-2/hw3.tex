\documentclass[11pt]{article}
\usepackage{amsmath,amssymb,amsthm}
\addtolength{\evensidemargin}{-.5in}
\addtolength{\oddsidemargin}{-.5in}
\addtolength{\textwidth}{0.8in}
\addtolength{\textheight}{0.8in}
\addtolength{\topmargin}{-.4in}
\newtheoremstyle{quest}{\topsep}{\topsep}{}{}{\bfseries}{}{ }{\thmname{#1}\thmnote{ #3}.}
\theoremstyle{quest}
\newtheorem*{definition}{Definition}
\newtheorem*{theorem}{Theorem}
\newtheorem*{question}{Question}
\newtheorem*{exercise}{Exercise}
\newtheorem*{challengeproblem}{Challenge Problem}
\newcommand{\name}{%%%%%%%%%%%%%%%%%%
%%%%%%%%%%%%%%%%%%%%%%%%%%%%%%
%%%%%%%%%%%%%%%%%%%%%%%%%%%%%%
%% put your name here, so we know who to give credit to %%
Dan Sokolsky
}%%%%%%%%%%%%%%%%%%%%%%%%%%%%%%
\newcommand{\hw}{%%%%%%%%%%%%%%%%%%%%
%% and which homework assignment is it? %%%%%%%%%
%% put the correct number below              %%%%%%%%%
%%%%%%%%%%%%%%%%%%%%%%%%%%%%%%
2
}
%%%%%%%%%%%%%%%%%%%%%%%%%%%%%%
%%%%%%%%%%%%%%%%%%%%%%%%%%%%%%
%%%%%%%%%%%%%%%%%%%%%%%%%%%%%%
\title{\vspace{-50pt}
\Huge \name
\\\vspace{20pt}
\huge Linear Algebra II\hfill Homework \hw}
\author{}
\date{}
\pagestyle{myheadings}
\markright{\name\hfill Homework \hw\qquad\hfill}

%% If you want to define a new command, you can do it like this:
\newcommand{\Q}{\mathbb{Q}}
\newcommand{\R}{\mathbb{R}}
\newcommand{\Z}{\mathbb{Z}}
\newcommand{\C}{\mathbb{C}}

%% If you want to use a function like ''sin'' or ''cos'', you can do it like this
%% (we probably won't have much use for this)
% \DeclareMathOperator{\sin}{sin}   %% just an example (it's already defined)


\begin{document}
\maketitle

\begin{question}[1]
Prove the minimal polynomial of the $n \times n$ matrix $A$, $m_A$, must include all distinct eigenvalues of $A$.
\end{question}
\begin{proof}
$(\lambda, v)$ an eigen-pair of $A \implies m(\lambda)v = m(A)v = 0 \cdot v = 0$, since $A^k v = \lambda^k v$. Since $v \ne 0$, it follows that $m(\lambda) = 0$.
\end{proof}
\begin{question}[2]
Given that the distinct roots of $m_A$ are the distinct eigenvalues of the $n \times n$ matrix $A$, show $-$
\\(a) $d_j \le m_j$ where $m_j$ is the algebraic multiplicity of the eigenvalue $a_j$, and $d_j$ is the multiplicity of $a_j$ as a root of  $m_A$.
\\(b) $d_j$ is the index of the eigenvalue $a_j$.
\end{question}
\begin{proof}
(a) Suppose $d_j > m_j$. Then $(t - a_j)^{d_j} \nmid (t - a_j)^{m_j}$. Thus $m_A \nmid p_A$, a contradiction.
\\(b) Suppose $m_A(t) = \prod_j(t-a_j)^{d_j}$. Define $N_j := N_{(A-a_jI)^{d_j}}$. Then $(t-a_i)^{d_i}$ and $(t-a_j)^{d_j}$ are coprime for $i \ne j$, and therefore, $\mathbb{C}^n = N_0 = N_{m_A(A)} = \bigoplus_j N_j$. Thus $0 = m_A(A) \restriction_{N_j} = (A - a_jI)^{d_j} \restriction_{N_j} \iff (A - a_jI)^{d_j} = 0 \iff d_j$ is the index of the eigenvalue $a_j$. Notice that $d_j$ is the minimal such power that satisfies our requirements.
\end{proof}

\end{document}
