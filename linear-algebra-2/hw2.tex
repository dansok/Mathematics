\documentclass[11pt]{article}
\usepackage{amsmath,amssymb,amsthm}
\addtolength{\evensidemargin}{-.5in}
\addtolength{\oddsidemargin}{-.5in}
\addtolength{\textwidth}{0.8in}
\addtolength{\textheight}{0.8in}
\addtolength{\topmargin}{-.4in}
\newtheoremstyle{quest}{\topsep}{\topsep}{}{}{\bfseries}{}{ }{\thmname{#1}\thmnote{ #3}.}
\theoremstyle{quest}
\newtheorem*{definition}{Definition}
\newtheorem*{theorem}{Theorem}
\newtheorem*{question}{Question}
\newtheorem*{exercise}{Exercise}
\newtheorem*{challengeproblem}{Challenge Problem}
\newcommand{\name}{%%%%%%%%%%%%%%%%%%
%%%%%%%%%%%%%%%%%%%%%%%%%%%%%%
%%%%%%%%%%%%%%%%%%%%%%%%%%%%%%
%% put your name here, so we know who to give credit to %%
Dan Sokolsky
}%%%%%%%%%%%%%%%%%%%%%%%%%%%%%%
\newcommand{\hw}{%%%%%%%%%%%%%%%%%%%%
%% and which homework assignment is it? %%%%%%%%%
%% put the correct number below              %%%%%%%%%
%%%%%%%%%%%%%%%%%%%%%%%%%%%%%%
2
}
%%%%%%%%%%%%%%%%%%%%%%%%%%%%%%
%%%%%%%%%%%%%%%%%%%%%%%%%%%%%%
%%%%%%%%%%%%%%%%%%%%%%%%%%%%%%
\title{\vspace{-50pt}
\Huge \name
\\\vspace{20pt}
\huge Linear Algebra II\hfill Homework \hw}
\author{}
\date{}
\pagestyle{myheadings}
\markright{\name\hfill Homework \hw\qquad\hfill}

%% If you want to define a new command, you can do it like this:
\newcommand{\Q}{\mathbb{Q}}
\newcommand{\R}{\mathbb{R}}
\newcommand{\Z}{\mathbb{Z}}
\newcommand{\C}{\mathbb{C}}

%% If you want to use a function like ''sin'' or ''cos'', you can do it like this
%% (we probably won't have much use for this)
% \DeclareMathOperator{\sin}{sin}   %% just an example (it's already defined)


\begin{document}
\maketitle

\begin{question}[1]
Give a straightforward proof of Theorem 1'
\end{question}
\begin{proof}
  For the sake of contradiction, suppose $h = \sum_{i \le m} c_i h_i$, while $a \ne a_i$ for any $i$. Then,
  $$\sum_{i \le m} a (c_i h_i) = a \sum_{i \le m} c_i h_i = ah = Ah = A \sum_{i \le m} c_i h_i = \sum_{i \le m} c_i (A h_i) = \sum_{i \le m} c_i (a_i h_i) = \sum_{i \le m} a_i (c_i h_i)$$ $\iff a = a_i$ for all $i \le m$, a gross contradiction to our assumption. Hence, $h$ is linearly independent from $\{h_i\}_{i \le m}$, as required.
\end{proof}

\begin{question}[2]
Is $\{h_i\}_{i \le m}$ required to be linearly independent?
\end{question}
\begin{proof}
  No. Indeed, nowhere have we assumed that $\{h_i\}_{i \le m}$ is linearly independent in the proof of $(1)$.
\end{proof}

\end{document}
