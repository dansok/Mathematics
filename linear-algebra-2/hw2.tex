\documentclass[11pt]{article}
\usepackage{amsmath,amssymb,amsthm}
\addtolength{\evensidemargin}{-.5in}
\addtolength{\oddsidemargin}{-.5in}
\addtolength{\textwidth}{0.8in}
\addtolength{\textheight}{0.8in}
\addtolength{\topmargin}{-.4in}
\newtheoremstyle{quest}{\topsep}{\topsep}{}{}{\bfseries}{}{ }{\thmname{#1}\thmnote{ #3}.}
\theoremstyle{quest}
\newtheorem*{definition}{Definition}
\newtheorem*{theorem}{Theorem}
\newtheorem*{question}{Question}
\newtheorem*{exercise}{Exercise}
\newtheorem*{challengeproblem}{Challenge Problem}
\newcommand{\name}{%%%%%%%%%%%%%%%%%%
%%%%%%%%%%%%%%%%%%%%%%%%%%%%%%
%%%%%%%%%%%%%%%%%%%%%%%%%%%%%%
%% put your name here, so we know who to give credit to %%
Dan Sokolsky
}%%%%%%%%%%%%%%%%%%%%%%%%%%%%%%
\newcommand{\hw}{%%%%%%%%%%%%%%%%%%%%
%% and which homework assignment is it? %%%%%%%%%
%% put the correct number below              %%%%%%%%%
%%%%%%%%%%%%%%%%%%%%%%%%%%%%%%
2
}
%%%%%%%%%%%%%%%%%%%%%%%%%%%%%%
%%%%%%%%%%%%%%%%%%%%%%%%%%%%%%
%%%%%%%%%%%%%%%%%%%%%%%%%%%%%%
\title{\vspace{-50pt}
\Huge \name
\\\vspace{20pt}
\huge Linear Algebra II\hfill Homework \hw}
\author{}
\date{}
\pagestyle{myheadings}
\markright{\name\hfill Homework \hw\qquad\hfill}

%% If you want to define a new command, you can do it like this:
\newcommand{\Q}{\mathbb{Q}}
\newcommand{\R}{\mathbb{R}}
\newcommand{\Z}{\mathbb{Z}}
\newcommand{\C}{\mathbb{C}}

%% If you want to use a function like ''sin'' or ''cos'', you can do it like this
%% (we probably won't have much use for this)
% \DeclareMathOperator{\sin}{sin}   %% just an example (it's already defined)


\begin{document}
\maketitle

\begin{question}[1]
Give a proof of the uniqueness of the decomposition $N_{pq} = N_p + N_q$ using formula $(31)$ on page $70$.
\end{question}
\begin{proof}
Let $x \in N_p \cap N_q$. Then $p(A)x = q(A)x = 0$. By formula $(31)$, there exist polynomials $a, b$, such that $x = a(A)p(A)x + b(A)q(A)x = a(A) \cdot 0 + b(A) \cdot 0 = 0$. Thus, $N_p \cap N_q = \{0\}$. Therefore, the decomposition is in fact a direct sum. I.e., $N_{pq} = N_p \oplus N_q$.
\end{proof}
\begin{question}[2]
Let $A$ be $n \times n$. Let $A^k = 0; A^{k-1} \ne 0$ for some $k > 0$.
\\(a) Show the eigenvalues of $A$ are all $0$.
\\(b) Show there exists $x \in \mathbb{C}^n$ such that $\{x, Ax, \ldots, A^{k-1}x\}$ is linearly independent.
\\(c) Find an eigenvector of $A$ derived from $x$.
\end{question}
\begin{proof}
Let $A$ be $n \times n$. Let $A^k = 0; A^{k-1} \ne 0$ for some $k > 0$.
\\(a) $Ax = cx \implies 0 = A^k x = c^k x \iff c^k = 0 \iff c = 0$, since $x \ne 0$ as an eigenvector.
\\ (b) Define $N_j := \mathbf{Null}(A^j)$, as before. Note, $A^i = 0$ for all $i \ge k$. Let $x$ be such that $A^{k-1}x \ne 0$. Then for all $j < k-1$, $A^j x \ne 0$ either, since $N_j \subset N_{k-1}$. Now, keeping with the convention $A^0 = I$, suppose $\sum_{i = 0}^{k-1} c_i A^i x = 0$. We show how to pick off the first term of the series iteratively, thereby showing that $c_i = 0$ by induction. To this end, consider
$$0 = A^{k-1}(0) = A^{k-1}(\sum_{i = 0}^{k-1} c_i A^i x) = \sum_{i = 0}^{k-1} c_i A^{i+k-1} x = c_0 A^{k-1} x$$
$\iff c_0 = 0$ since $A^{k-1}x \ne 0$, and $A^i = 0$ for all $i \ge k$. So we reduce the series $-$ $0 = \sum_{i = 0}^{k-1} c_i A^i x = \sum_{i = 1}^{k-1} c_i A^i x$; where we can apply the same argument to the latter, using $A^{k-2}$ this time. Continuing in this way, and applying $A^{k-1-j}$ at each step $ 0 \le j \le k-1$. We conclude $c_i = 0$ for all $ 0 \le i \le k-1$ by induction. Thus, $\{x, Ax, \ldots, A^{k-1}x\}$ is linearly independent.
\\(c) Take $x$ such that $A^{k-1} x \ne 0$. Then $A(A^{k-1} x) = A^k x = 0$. Thus $A^{k-1} x$ is an eigenvector of A.
\end{proof}
\begin{question}[3]
Let $A$ be $n \times n$. Let $A^k = 0; A^{k-1} \ne 0$ for some $k > 0$, so that all eigenvalues are $0$. Define $N_i := \mathbf{Null}(A^i)$.
\\(a) What is the number of linearly independent $x$ such that $\{x, Ax, \ldots, A^{k-1}x\}$ is linearly independent?
\\(b) By regarding $x$ as an equivalent class, identify the quotient space spanned by these linearly independent $x$. 
\end{question}
\begin{proof}
(a) As we have shown in question (2)(b), $x \in \mathbb{R}^n$, such that $A^{k-1} x \ne 0 \implies \{x, Ax, \ldots, A^{k-1}x\}$ is linearly independent. That is, $x \in \mathbb{R}^n \setminus N_{k-1} \implies \{x, Ax, \ldots, A^{k-1}x\}$ is linearly independent. By the Rank-Nullity theorem, the number of such linearly independent $x$ is $\dim((\mathbb{R}^n \setminus N_{k-1}) \cup \{0\}) = Rank(A^{k-1})$.
\\(b) Define $Z := (\mathbb{R}^n \setminus N_{k-1}) \cup \{0\}$. Then, $Z \subset \mathbb{R}^n$ is a linear subspace; and, since for $x \in \mathbb{R}^n$, $A^{k-1}x \ne 0 \implies x \in Z$, and $A^{k-1}x = 0 \implies x \in N_{k-1}$, and $Z \cap N_{k-1} = \{0\}$, naturally $\mathbb{R}^n = Z \oplus N_{k-1}$ (since this is a direct sum it's already clear $N_{k-1}$ is the quotient since the dimensions must line up). Once again by the Rank-Nullity theorem, $$\dim(Z) = Rank(A^{k-1}) = \dim(\mathbb{R}^n) - Nullity(A^{k-1}) = \dim(\mathbb{R}^n) - \dim(N_{k-1})$$
$$\iff \dim(N_{k-1}) = \dim(\mathbb{R}^n) - \dim(Z) = \dim(\mathbb{R}^n / Z)$$
That is, $\mathbb{R}^n / Z \cong N_{k-1}$.
\end{proof}
\begin{question}[4]
Let $A: X \rightarrow X$. Define $N_j := \mathbf{Null}(A^j)$.
\\(a) Prove $N_j \subset N_{j+1}$
\\(b) Prove $N_k = N_{k+1} \implies N_i = N_k$ for all $i > k$
\end{question}
\begin{proof}
  Let $A, N_j$ be as in the statement of the problem.
  \\(a) $x \in N_j \implies A^{j+1}x = A(A^j x) = A(0) = 0$.
  \\(b) $x \in N_{k+2} \implies 0 = A^{k+2}x = A^{k+1}(Ax) = A^k(Ax) = A^{k+1}x$.
\end{proof}
\begin{question}[5]
Let $A: X \rightarrow X$. Define $N_j := \mathbf{Null}(A^j)$.
\\(a) Prove $A(N_{i+1}) \subset N_i, i \ge 1$
\\(b) Prove $A(N_{i+1}) \subset N_{i-1} \implies N_i = N_{i+1}$
\end{question}
\begin{proof}
Let $A, N_j$ be as in the statement of the problem.
  \\(a) $y \in A(N_{i+1}) \implies y = A(x)$, and $A^i(y) = A^i(A(x)) = A^{i+1}(x) = 0$.
  \\(b) $x \in N_{i+1} \implies A(x) \in A(N_{i+1}) \subset N_{i-1}$. Thus, $0 = A^{i-1}(A(x)) = A^i(x)$.
\end{proof}

\end{document}
