\documentclass[11pt]{article}
\usepackage{amsmath,amssymb,amsthm}
\addtolength{\evensidemargin}{-.5in}
\addtolength{\oddsidemargin}{-.5in}
\addtolength{\textwidth}{0.8in}
\addtolength{\textheight}{0.8in}
\addtolength{\topmargin}{-.4in}
\newtheoremstyle{quest}{\topsep}{\topsep}{}{}{\bfseries}{}{ }{\thmname{#1}\thmnote{ #3}.}
\theoremstyle{quest}
\newtheorem*{definition}{Definition}
\newtheorem*{theorem}{Theorem}
\newtheorem*{question}{Question}
\newtheorem*{exercise}{Exercise}
\newtheorem*{challengeproblem}{Challenge Problem}
\newcommand{\name}{%%%%%%%%%%%%%%%%%%
%%%%%%%%%%%%%%%%%%%%%%%%%%%%%%
%%%%%%%%%%%%%%%%%%%%%%%%%%%%%%
%% put your name here, so we know who to give credit to %%
Dan Sokolsky
}%%%%%%%%%%%%%%%%%%%%%%%%%%%%%%
\newcommand{\hw}{%%%%%%%%%%%%%%%%%%%%
%% and which homework assignment is it? %%%%%%%%%
%% put the correct number below              %%%%%%%%%
%%%%%%%%%%%%%%%%%%%%%%%%%%%%%%
2
}
%%%%%%%%%%%%%%%%%%%%%%%%%%%%%%
%%%%%%%%%%%%%%%%%%%%%%%%%%%%%%
%%%%%%%%%%%%%%%%%%%%%%%%%%%%%%
\title{\vspace{-50pt}
\Huge \name
\\\vspace{20pt}
\huge Linear Algebra II\hfill Homework \hw}
\author{}
\date{}
\pagestyle{myheadings}
\markright{\name\hfill Homework \hw\qquad\hfill}

%% If you want to define a new command, you can do it like this:
\newcommand{\Q}{\mathbb{Q}}
\newcommand{\R}{\mathbb{R}}
\newcommand{\Z}{\mathbb{Z}}
\newcommand{\C}{\mathbb{C}}

%% If you want to use a function like ''sin'' or ''cos'', you can do it like this
%% (we probably won't have much use for this)
% \DeclareMathOperator{\sin}{sin}   %% just an example (it's already defined)


\begin{document}
\maketitle

\begin{question}[1]
Let $\{(a_i, h_i)\}_{i \le m}$ be eigen-pairs of the matrix $A$, so that $Ah_i = a_i h_i, 0 \ne h_i \in \mathbb{R}^n$; and $\{h_i\}_{i \le m}$ is linearly independent. If $a \ne a_i$ for all $i \le m$, then the eigenvector corresponding to $a$, $h \in \mathbb{R}^n$, is linearly independent from $\{h_i\}_{i \le m}$.
\end{question}
\begin{proof}
  For the sake of contradiction, suppose $h = \sum_{i \le m} c_i h_i$, while $a \ne a_i$ for any $i \le m$. Then,
  $$\sum_{i \le m} a (c_i h_i) = a \sum_{i \le m} c_i h_i = ah = Ah = A \sum_{i \le m} c_i h_i = \sum_{i \le m} c_i (A h_i) = \sum_{i \le m} c_i (a_i h_i) = \sum_{i \le m} a_i (c_i h_i)$$ $\iff a = a_i$ for all $i \le m$, a gross contradiction to our assumption. Hence, $h$ is linearly independent from $\{h_i\}_{i \le m}$, as required.
\end{proof}

\begin{question}[2]
Is $\{h_i\}_{i \le m}$ required to be linearly independent?
\end{question}
\begin{proof}
  No. Indeed, nowhere have we assumed that $\{h_i\}_{i \le m}$ is linearly independent in the proof of $(1)$.
\end{proof}

\begin{question}[3]
Let $A: X \rightarrow X$. Define $N_j := \mathbf{Null}(A^j)$.
\\(a) Prove $N_j \subset N_{j+1}$
\\(b) Prove $N_k = N_{k+1} \implies N_i = N_k$ for all $i > k$
\end{question}
\begin{proof}
  Let $A, N_j$ be as in the statement of the problem.
  \\(a) Let $x \in N_j$. Then $A^j(x) = 0$. Then $A^{j+1}(x) = A(A^j(x)) = A(0) = 0$, since A is linear. I.e., $x \in N_{j+1}$. Thus $N_j \subset N_{j+1}$, as required.
  \\(b) Let $N_k = N_{k+1}$. Let $x \in N_{k+2}$. Then, $0 = A^{k+2}(x) = A^{k+1}(A(x))$. I.e., $A(x) \in N_{k+1} = N_k$. Thus, $0 = A^k(A(x)) = A^{k+1}(x)$. That is, $x \in N_{k+1}$. I.e., $N_{k+2} \subset N_{k+1}$. By (a), $N_{k+1} \subset N_{k+2}$. Therefore, $N_{k+2} = N_{k+1}$. Now, using the same argument, $N_{k+2} = N_{k+1}$ implies $N_{k+3} = N_{k+2}$ (since $k$ is arbitrary). By induction, we conclude $N_i = N_k$ for all $i > k$, as required.
\end{proof}

\begin{question}[4]
Let $A: X \rightarrow X$. Define $N_j := \mathbf{Null}(A^j)$.
\\(a) Prove $A(N_{i+1}) \subset N_i, i \ge 1$
\\(b) Prove $A(N_{i+1}) \subset N_{i-1} \implies N_i = N_{i+1}$
\end{question}
\begin{proof}
Let $A, N_j$ be as in the statement of the problem.
  \\(a) First note, $i \ge 1 \iff i-1 \ge 0$. Let $y \in A(N_{i+1})$. Then, $ y = A(x)$, for some $x \in N_{i+1}$, and $A^i(y) = A^i(A(x)) = A^{i+1}(x) = 0$. Hence, $y \in N_i$. I.e., $A(N_{i+1}) \subset N_i$, as required.
  \\(b) Suppose $A(N_{i+1}) \subset N_{i-1}$.
\end{proof}

\end{document}
