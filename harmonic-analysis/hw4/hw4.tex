\documentclass[12pt]{article}

\usepackage[margin=1in]{geometry}
\usepackage{amsmath, amsthm, amssymb, amsfonts, mathrsfs, unicode-math}
\usepackage{esint}

\newcommand{\N}{\mathbb{N}}
\newcommand{\Z}{\mathbb{Z}}

\newenvironment{theorem}[2][Theorem]{\begin{trivlist}
\item[\hskip \labelsep {\bfseries #1}\hskip \labelsep {\bfseries #2.}]}{\end{trivlist}}
\newenvironment{lemma}[2][Lemma]{\begin{trivlist}
\item[\hskip \labelsep {\bfseries #1}\hskip \labelsep {\bfseries #2.}]}{\end{trivlist}}
\newenvironment{exercise}[2][Exercise]{\begin{trivlist}
\item[\hskip \labelsep {\bfseries #1}\hskip \labelsep {\bfseries #2.}]}{\end{trivlist}}
\newenvironment{problem}[2][Problem]{\begin{trivlist}
\item[\hskip \labelsep {\bfseries #1}\hskip \labelsep {\bfseries #2.}]}{\end{trivlist}}
\newenvironment{question}[2][Question]{\begin{trivlist}
\item[\hskip \labelsep {\bfseries #1}\hskip \labelsep {\bfseries #2.}]}{\end{trivlist}}
\newenvironment{corollary}[2][Corollary]{\begin{trivlist}
\item[\hskip \labelsep {\bfseries #1}\hskip \labelsep {\bfseries #2.}]}{\end{trivlist}}


\def\Xint#1{\mathchoice
{\XXint\displaystyle\textstyle{#1}}%
{\XXint\textstyle\scriptstyle{#1}}%
{\XXint\scriptstyle\scriptscriptstyle{#1}}%
{\XXint\scriptscriptstyle\scriptscriptstyle{#1}}%
\!\int}
\def\XXint#1#2#3{{\setbox0=\hbox{$#1{#2#3}{\int}$ }
\vcenter{\hbox{$#2#3$ }}\kern-.6\wd0}}
\def\ddashint{\Xint=}
\def\dashint{\Xint-}

% -----------------------------------------
% #########################################
% -----------------------------------------
% INTERMISSION
% -----------------------------------------
% #########################################
% -----------------------------------------

\begin{document}

\title{Harmonic Analysis}
\title{Homework #4}
\author{Dan Sokolsky}

\maketitle

\begin{exercise}{1}
\end{exercise}

\begin{proof}
  Let $\{Q_j\}$ be the cubes defined in the proof of the Calderón–Zygmund lemma. Let $S = \{x \in [0, 1)^n\ |\ M_4f(x) > \lambda\}$, where $x_0 = 0$ in the definition
  $$M_4f(x) = \sup_{Q \in \mathfrak{D}\ :\ Q \in [0, 1]^n} \dashint_Q |f|$$
  $(\subseteq)$ Let $Q_j = [a, b]^n$. Let $\tilde{Q}_j = [a, b)^n$. Then, $Q_j, \tilde{Q}_j$ only differ on a set of measure 0. By construction,
  $$\dashint_{\tilde{Q}_j} |f| = \dashint_{Q_j} |f| > \lambda$$
  so that $M_4f(x) > \lambda$ for all $x \in Q_j$. Therefore, $Q_j \subseteq S$ (for a.e. $x \in Q_j$). Thus, $\cup Q_j \subseteq S$ up to a set of measure 0.\\
  $(\supseteq)$ Let $x \in S$. Then there exists a dyadic cube $Q \subseteq [0, 1)^n$ such that
  $$\dashint_{Q} |f| > \lambda$$
  So that $Q = Q_j$ for some $j$. Thus $S \subseteq \cup Q_j$. Therefore, $S = \cup Q_j$ up to a set of measure 0.
\end{proof}

\begin{exercise}{2}
\end{exercise}

\begin{proof}
  $(a) (i)$ Since $\lambda > 0$ and $\mu\{|f| > \lambda\} \ge 0$, we have $\|f\|_{L^{p, \infty}}\ge 0$.\\
  $(ii)$
  $$\|kf\|_{L^{p, \infty}} = \sup_{\lambda > 0} \lambda \cdot \mu\{|kf| > \lambda\} = \sup_{\lambda > 0} \lambda \cdot \mu\{|f| > \frac{\lambda}{|k|}\} =$$
  $$\sup_{|k|\lambda > 0} |k|\lambda \cdot \mu\{|f| > \frac{|k|\lambda}{|k|}\} = \sup_{|k|\lambda > 0} |k|\lambda \cdot \mu\{|f| > \frac{|k|\lambda}{|k|}\} =
  |k| \cdot \sup_{|k|\lambda > 0} \lambda \cdot \mu\{|f| > \lambda\} =$$
  $$|k| \cdot \sup_{\lambda > 0} \lambda \cdot \mu\{|f| > \lambda\} = |k| \cdot \|f\|_{L^{p, \infty}}$$
  $(iii)$
  $$\|f + g\|_{L^{p, \infty}} = \sup_{\lambda > 0} \lambda \cdot \mu\{|f + g| > \lambda\} \le \sup_{\lambda > 0} \lambda \cdot (\mu\{|f| > \frac{\lambda}{2}\} + \mu\{|g| > \frac{\lambda}{2}\}) \le$$
  $$\sup_{\lambda > 0} \lambda \cdot (\mu\{|f| > \frac{\lambda}{2}\}) + \sup_{\lambda > 0} \lambda \cdot (\mu\{|g| > \lambda\}) =$$
  $$\sup_{2\lambda > 0} 2\lambda \cdot (\mu\{|f| > \lambda\}) + \sup_{2\lambda > 0} 2\lambda \cdot (\mu\{|g| > \lambda\}) =$$
  $$2\sup_{\lambda > 0} \lambda \cdot (\mu\{|f| > \lambda\}) + 2\sup_{\lambda > 0} \lambda \cdot (\mu\{|g| > \lambda\}) = 2 (\|f\|_{L^{p, \infty}} + \|g\|_{L^{p, \infty}})$$
  $(b)$
  $$\int_E |f| \ge \int_E \lambda = \lambda \mu(E)$$
  So that,
  $$\mu(E)^{\frac{1}{p}-1} \int_E |f| \ge \mu(E)^{\frac{1}{p}-1} \cdot \lambda \mu(E) = \lambda \cdot \mu(E)^{\frac{1}{p}}$$
  Thus,
  $$\|f\|' = \sup_{E: 0 < \mu(E) < \infty} \mu(E)^{\frac{1}{p}-1} \int_E |f| \ge \sup_{\lambda > 0} \lambda \mu\{|f| > \lambda\}^{\frac{1}{p}} = \|f\|_{L^{p, \infty}}$$
  $(c)$
  $$\int_E |f(x)| d\mu = \int_E \Big(\int_0^{|f(x)|} d\lambda \Big) du = \int_E \mu\{|f(x)| > \lambda\} d\lambda = \int_0^{\infty} \mu\{x \in E\ |\ |f(x)| > \lambda\} d\lambda \le$$
  $$\int_0^{\mu(E)^{-\frac{1}{p}}\|f\|_{L^{p, \infty}}} \mu(E) d\lambda + \int_{\mu(E)^{-\frac{1}{p}}\|f\|_{L^{p, \infty}}}^{\infty} \mu\{|f| > \lambda\} d\lambda le$$
  $$\int_0^{\mu(E)^{-\frac{1}{p}}\|f\|_{L^{p, \infty}}} \mu(E) d\lambda + \int_{\mu(E)^{-\frac{1}{p}}\|f\|_{L^{p, \infty}}}^{\infty} \dfrac{\|f\|_{L^{p, \infty}}}{\lambda^p} d\lambda = $$
  $$\mu(E)^{-\frac{1}{p}}\|f\|_{L^{p, \infty}}\mu(E) + \|f\|_{L^{p, \infty}}^p \cdot \Big[\dfrac{\lambda^{1-p}}{1-p}\Big]_{\mu(E)^{-\frac{1}{p}}\|f\|_{L^{p, \infty}}}^{\infty} =$$
  $$\mu(E)^{\frac{p-1}{p}} \|f\|_{L^{p, \infty}} \cdot \dfrac{p}{p-1} \iff$$
  $$\mu(E)^{\frac{1}{p} - 1} \int_E |f| d\mu \le \dfrac{p}{p-1} \|f\|_{L^{p, \infty}}$$
  For all Borel sets $E$. Therefore,
  $$\|f\|' = \sup_{E: 0 < \mu(E) < \infty} \mu(E)^{\frac{1}{p} - 1} \int_E |f| d\mu \le \dfrac{p}{p-1} \|f\|_{L^{p, \infty}}$$
  $(d)$ We have
  $$\|f\|_{L^{p, \infty}} \le \|f\|' \le \dfrac{p}{p-1} \|f\|_{L^{p, \infty}}$$
  Thus the two norms are equivalent.
\end{proof}

\begin{exercise}{3}
\end{exercise}

\begin{proof}
\end{proof}

\end{document}
