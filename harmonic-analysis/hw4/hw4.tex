\documentclass[12pt]{article}

\usepackage[margin=1in]{geometry}
\usepackage{amsmath, amsthm, amssymb, amsfonts, mathrsfs, unicode-math}
\usepackage{esint}

\newcommand{\N}{\mathbb{N}}
\newcommand{\Z}{\mathbb{Z}}

\newenvironment{theorem}[2][Theorem]{\begin{trivlist}
\item[\hskip \labelsep {\bfseries #1}\hskip \labelsep {\bfseries #2.}]}{\end{trivlist}}
\newenvironment{lemma}[2][Lemma]{\begin{trivlist}
\item[\hskip \labelsep {\bfseries #1}\hskip \labelsep {\bfseries #2.}]}{\end{trivlist}}
\newenvironment{exercise}[2][Exercise]{\begin{trivlist}
\item[\hskip \labelsep {\bfseries #1}\hskip \labelsep {\bfseries #2.}]}{\end{trivlist}}
\newenvironment{problem}[2][Problem]{\begin{trivlist}
\item[\hskip \labelsep {\bfseries #1}\hskip \labelsep {\bfseries #2.}]}{\end{trivlist}}
\newenvironment{question}[2][Question]{\begin{trivlist}
\item[\hskip \labelsep {\bfseries #1}\hskip \labelsep {\bfseries #2.}]}{\end{trivlist}}
\newenvironment{corollary}[2][Corollary]{\begin{trivlist}
\item[\hskip \labelsep {\bfseries #1}\hskip \labelsep {\bfseries #2.}]}{\end{trivlist}}


\def\Xint#1{\mathchoice
{\XXint\displaystyle\textstyle{#1}}%
{\XXint\textstyle\scriptstyle{#1}}%
{\XXint\scriptstyle\scriptscriptstyle{#1}}%
{\XXint\scriptscriptstyle\scriptscriptstyle{#1}}%
\!\int}
\def\XXint#1#2#3{{\setbox0=\hbox{$#1{#2#3}{\int}$ }
\vcenter{\hbox{$#2#3$ }}\kern-.6\wd0}}
\def\ddashint{\Xint=}
\def\dashint{\Xint-}

% -----------------------------------------
% #########################################
% -----------------------------------------
% INTERMISSION
% -----------------------------------------
% #########################################
% -----------------------------------------

\begin{document}

\title{Harmonic Analysis}
\title{Homework #4}
\author{Dan Sokolsky}

\maketitle

\begin{exercise}{1}
\end{exercise}

\begin{proof}
  Let $\{Q_j\}$ be the cubes defined in the proof of the Calderón–Zygmund lemma. Let $S = \{x \in [0, 1)^n\ |\ M_4f(x) > \lambda\}$, where $x_0 = 0$ in the definition
  $$M_4f(x) = \sup_{Q \in \mathfrak{D}\ :\ Q \in [0, 1]^n} \dashint_Q |f|$$
  $(\subseteq)$ Let $Q_j = [a, b]^n$. Let $\tilde{Q}_j = [a, b)^n$. Then, $Q_j, \tilde{Q}_j$ only differ on a set of measure 0. By construction,
  $$\dashint_{\tilde{Q}_j} |f| = \dashint_{Q_j} |f| > \lambda$$
  so that $M_4f(x) > \lambda$ for all $x \in Q_j$. Therefore, $Q_j \subseteq S$ (for a.e. $x \in Q_j$). Thus, $\cup Q_j \subseteq S$ up to a set of measure 0.\\
  $(\supseteq)$ Let $x \in S$. Then there exists a dyadic cube $Q \subseteq [0, 1)^n$ such that
  $$\dashint_{Q} |f| > \lambda$$
  So that $Q = Q_j$ for some $j$. Thus $S \subseteq \cup Q_j$. Therefore, $S = \cup Q_j$ up to a set of measure 0.
\end{proof}

\begin{exercise}{2}
\end{exercise}

\begin{proof}
\end{proof}

\begin{exercise}{3}
\end{exercise}

\begin{proof}
\end{proof}

\end{document}
