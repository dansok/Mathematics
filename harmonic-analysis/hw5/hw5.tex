\documentclass[12pt]{article}

\usepackage[margin=1in]{geometry}
\usepackage{amsmath, amsthm, amssymb, amsfonts, mathrsfs, unicode-math}

\newcommand{\N}{\mathbb{N}}
\newcommand{\Z}{\mathbb{Z}}

\newenvironment{theorem}[2][Theorem]{\begin{trivlist}
\item[\hskip \labelsep {\bfseries #1}\hskip \labelsep {\bfseries #2.}]}{\end{trivlist}}
\newenvironment{lemma}[2][Lemma]{\begin{trivlist}
\item[\hskip \labelsep {\bfseries #1}\hskip \labelsep {\bfseries #2.}]}{\end{trivlist}}
\newenvironment{exercise}[2][Exercise]{\begin{trivlist}
\item[\hskip \labelsep {\bfseries #1}\hskip \labelsep {\bfseries #2.}]}{\end{trivlist}}
\newenvironment{problem}[2][Problem]{\begin{trivlist}
\item[\hskip \labelsep {\bfseries #1}\hskip \labelsep {\bfseries #2.}]}{\end{trivlist}}
\newenvironment{question}[2][Question]{\begin{trivlist}
\item[\hskip \labelsep {\bfseries #1}\hskip \labelsep {\bfseries #2.}]}{\end{trivlist}}
\newenvironment{corollary}[2][Corollary]{\begin{trivlist}
\item[\hskip \labelsep {\bfseries #1}\hskip \labelsep {\bfseries #2.}]}{\end{trivlist}}

% -----------------------------------------
% #########################################
% -----------------------------------------
% INTERMISSION
% -----------------------------------------
% #########################################
% -----------------------------------------

\begin{document}

\title{Harmonic Analysis}
\title{Homework #5}
\author{Dan Sokolsky}

\maketitle

\begin{exercise}{1}
\end{exercise}

\begin{proof}
  First, we claim $f_b \in L^1(\mathbb{R}^n)$, and $f_s \in L^r(\mathbb{R}^n)$ -
  $$\int |f_b| = \int |f_b|^p |f_b|^{1-p} \le C(p) \|f\|^p_p < \infty$$
  $$\int |f_s|^r = \int |f_s|^{r-p} |f_s|^p \le C(r, p) \|f\|^p_p < \infty$$
  Now, since $|Tf| \le |Tf_b| + |Tf_s|$, we have
  $\{|Tf| > t\} \subseteq \{|Tf_b| > \frac{t}{2}\} \cup \{|Tf_s| > \frac{t}{2}\}$, so that -

  $$\mu\{|Tf| > t\} \le \mu\{|Tf_b| > \frac{t}{2}\} + \mu\{|Tf_s| > \frac{t}{2}\} \le$$
  $$\dfrac{2A \|f\|_1}{t} \int |f_b| + \dfrac{2^rA^r \|f\|_r^r}{t^r} \int |f_s|^r$$
  Now,
  $$\int_0^\infty t^{q-1}t^{-1} \int_{|f| > t} |f| = \int_{\mathbb{R}^n} |f| \int_0^{|f|} t^{q-2} = \dfrac{1}{q-1} \int_{\mathbb{R}^n} |f||f|^{q-1} = \dfrac{\|f\|^q_q}{q-1}$$
  since $q > p > 1$, and
  $$\int_0^\infty t^{q-1}t^{-r} \int_{|f| \le t} |f|^r = \int_{\mathbb{R}^n} |f|^r \int_{|f|}^{\infty} t^{q-1-r} = \dfrac{1}{r-q} \int_{\mathbb{R}^n} |f|^r|f|^{q-r} = \dfrac{\|f\|^q_q}{r-q}$$
  since $q < r$. Altogether,
  $$\|Tf\|_q \le C \|f\|_q$$
\end{proof}

\begin{exercise}{2}
\end{exercise}

\begin{proof}
  $(a)$
  $$\int_{\mathbb{R}^n \setminus B_{2r}(0)} |K(x) - K(x-z)| dx = \int_{\mathbb{R}^n \setminus B_{2r}(0)} \Big| \int_{0}^{1} DK d\gamma \Big| dx =$$
  $$\int_{\mathbb{R}^n \setminus B_{2r}(0)} \int_{0}^{1} \Big|DK(t)(x-(1-t)z) \cdot z\ \Big| dt\ dx \le \int_{\mathbb{R}^n \setminus B_{2r}(0)} B |x|^{-n-1} |x - (x-z)| dx =$$
  $$B |z| \int_{\mathbb{R}^n \setminus B_{2r}(0)} |x|^{-n-1}dx = C(n)B$$\\
  $(b)$
  $$\int_{\mathbb{R}^n} \Big( K(x) - K(x - x_{\xi}) \Big) e^{-2 \pi i x \xi}\ dx = \int_{\mathbb{R}^n} K(x) e^{-2 \pi i x \xi}\ dx - \int_{\mathbb{R}^n} K(x-x_\xi) e^{-2 \pi i x \xi}\ dx =$$
  $$\hat{K}(\xi) - \int_{\mathbb{R}^n} K(x) e^{-2 \pi i (x + x_\xi) \xi}\ dx = \hat{K}(\xi) - e^{-i \pi} \int_{\mathbb{R}^n} K(x) e^{-2 \pi i x \xi}\ dx =$$
  $$\hat{K}(\xi) + \hat{K}(\xi) = 2 \hat{K}(\xi)$$\\
  $(c)$ By the  cancellation condition, we have
  $$\Big| \int_{B_\frac{1}{|\xi|}(0)} K(x)e^{-2 \pi i x \xi} \Big| = \Big| \int_{B_\frac{1}{|\xi|}(0)} K(x)e^{-2 \pi i x \xi} - \int_{B_\frac{1}{|\xi|}(0)} K(x) \Big| =$$
  $$\Big| \int_{B_\frac{1}{|\xi|}(0)} K(x)(e^{-2 \pi i x \xi} - 1) \Big| \le \int_{B_\frac{1}{|\xi|}(0)} |K(x)|\cdot |e^{-2 \pi i x \xi} - 1| \le 2\pi|\xi||x| \int_{B_\frac{1}{|\xi|}(0)} |K(x)| \le$$
  $$2\pi|\xi||x| A \int_{B_\frac{1}{|\xi|}(0)} |x|^{-n} = 2\pi|\xi| A \int_{B_\frac{1}{|\xi|}(0)} |x|^{-n+1} = C(n)A$$\\
  $(d)$
  $$\Big| \int_{B_\frac{1}{|\xi|}(0)} K(x-x_\xi)e^{-2 \pi i x \xi} \Big| = \Big| e^{i\pi}\int_{B_\frac{1}{|\xi|}(0)} K(x-x_\xi)e^{-2 \pi i x \xi} \Big| = \Big| \int_{B_\frac{1}{|\xi|}(0)} K(x-x_\xi)e^{-2 \pi i (x-x_{\xi}) \xi} \Big| =$$
  $$\Big| \int_{B_{\frac{1}{|\xi|}}(x_\xi)} K(x)e^{-2 \pi i x \xi} \Big|$$
  Since $|x_\xi| = \frac{1}{2|\xi|} < \frac{1}{|\xi|}$, we have that $B_{r}(0) \subseteq B_{\frac{1}{|\xi|}}(x_\xi)$ for some $0 < r < \frac{1}{|\xi|}$. Thus, as in $(c)$,
  $$|\int_{B_r(0)} K(x-x_\xi)e^{-2 \pi i x \xi}| \le 2\pi r A \int_{B_r(0)} |x|^{-n+1} \le 2\pi|\xi| A \int_{B_\frac{1}{|\xi|}(0)} |x|^{-n+1} = C(n)A$$
  Now,
  $$\Big| \int_{B_{\frac{1}{|\xi|}}(x_\xi)} K(x)e^{-2 \pi i x \xi} \Big| = \Big| \int_{B_{\frac{1}{|\xi|}}(x_\xi) \setminus B_r(0)} K(x)e^{-2 \pi i x \xi} + \int_{B_r(0)} K(x)e^{-2 \pi i x \xi} \Big| \le$$
  $$\Big| \int_{B_{\frac{1}{|\xi|}}(x_\xi) \setminus B_r(0)} K(x)e^{-2 \pi i x \xi}\Big| + \Big|\int_{B_r(0)} K(x)e^{-2 \pi i x \xi} \Big| \le$$ $$A\int_{B_{\frac{1}{|\xi|}}(x_\xi) \setminus B_r(0)} \frac{1}{|x|^n} + C(n)A = C + C(n)A \le C_1(n)A$$
  $(e)$
  $$\Big| 2\hat{K}(\xi) \Big| = \Big| \int_{\mathbb{R}^n} K(x) e^{-2 \pi i x \xi}\ dx \Big| = \Big|\int_{\mathbb{R}^n} \Big( K(x) - K(x - x_{\xi}) \Big) e^{-2 \pi i x \xi}\ dx \Big| \le$$
  $$\Big|\int_{\mathbb{R}^n \setminus B_\frac{1}{|\xi|}(0)} \Big( K(x) - K(x - x_{\xi}) \Big) e^{-2 \pi i x \xi}\ dx + \int_{B_\frac{1}{|\xi|}(0)} \Big( K(x) - K(x - x_{\xi}) \Big) e^{-2 \pi i x \xi}\ dx\Big| \le$$
  $$\int_{\mathbb{R}^n \setminus B_\frac{1}{|\xi|}(0)} \Big| K(x) - K(x - x_{\xi}) \Big|\ dx + \Big|\int_{B_\frac{1}{|\xi|}(0)} K(x)e^{-2 \pi i x \xi}\Big| + \Big| \int_{B_\frac{1}{|\xi|}(0)} K(x-x_\xi)e^{-2 \pi i x \xi} \Big| \le$$
  $$A + 2C(n)A \le C_1(n)A \iff \Big| \hat{K}(\xi) \Big| \le \frac{C_1(n)A}{2}$$
\end{proof}

\begin{exercise}{3}
\end{exercise}

\begin{proof}
$(a)$ $(i)$ $|K \mathbf{1}_{B_r \setminus B_\epsilon}(x)| \le |K(x)| \le A|x|^{-n}$\\
$(ii)$\\
If $r < \frac{R}{2}$,  we have
$$\int_{\mathbb{R}^n \setminus B_{2r}(0)} |K_{\epsilon, R}(x) - K_{\epsilon, R}(x-z)| dx \le \int_{\mathbb{R}^n \setminus B_{R}(0)} |K_{\epsilon, R}(x) - K_{\epsilon, R}(x-z)| dx \le$$
$$\int_{\mathbb{R}^n \setminus B_{R}(0)} \frac{1}{|x|^n} dx \le A \int_{B_R(0) \setminus B_{\frac{R}{2}}(0)} \frac{1}{|x-z|^n} dx =$$
$$A \int_{B_R(0) \setminus B_{\frac{R}{2}}(0)} \frac{1}{|x|^n} dx = C_1(n)A$$
If $\frac{R}{2} \le r \le R$,  we have
$$\int_{\mathbb{R}^n \setminus B_{2r}(0)} |K_{\epsilon, R}(x) - K_{\epsilon, R}(x-z)| dx = \int_{\mathbb{R}^n \setminus B_{2r}(0) \cap \{K(x) \ne 0\} \cap \{K(x-z) \ne 0\}} |K_{\epsilon, R}(x) - K_{\epsilon, R}(x-z)| dx \le$$
$$A \int_{B_{R}(0) \setminus B_{\frac{R}{2}}(0)} \frac{1}{|x|^n} dx + A \Big(\mu\{B_R(0) - B_\frac{R}{2}(0)\}\Big) \le C_2(n) A$$
If $r > R$,  we have
$$|x-z| \ge |x| - |z| \ge 2R - R$$
so that the region of integration falls outside the support of $K(x)$, and $K(x-z)$. Therefore,
$$\int_{\mathbb{R}^n \setminus B_{2r}(0)} |K_{\epsilon, R}(x) - K_{\epsilon, R}(x-z)| dx = 0$$
$(iii)$
$$\int_{B_s(0) \setminus B_r(0)} K = \int_{B_s(0)} K - \int_{B_r(0)} K = 0 - 0 = 0$$
$(b)$ We will prove $\|K_{\epsilon, R} * f\|_{L^2} \le \|\hat{K}_{\epsilon, R}\|_{L^{\infty}} \|f\|_{L^2}$. The general inequality $\|K_{\epsilon, R} * f\|_{L^2} \le C(n, p) A \|f\|_{L^p}$ then follows by the Marcinkiewicz Interpolation Theorem, the same way we proved the Calderon-Zygmund estimate. To  that end, by Plancheral's identity, we have -
$$\|K_{\epsilon, R} * f\|_{L^2} = \|\mathcal{F}(K_{\epsilon, R} * f)\|_{L^2} = \|\hat{K}_{\epsilon, R} \cdot \hat{f}\|_{L^2} \le \|\hat{K}_{\epsilon, R}\|_{L^{\infty}} \cdot \|\hat{f}\|_{L^2} = C(n) A \|\hat{f}\|_{L^2}$$
$(c)$ Since $f$ is smooth and has compact support, by the Extreme Value Theorem, $f$ achieves a minimum and a maximum on it's domain. Let $m, M$ be the minimal, and maximal values of $f$, respectively. WLOG, suppose $R \ge a$. Then,
$$|(K_{\epsilon, R} * f)(x)| = \Big| \int K_{\epsilon, R}(y) f(x-y) dy \Big| = \Big| \int_{|x| \ge a} K_{\epsilon, R}(y) f(x-y) dy + \int_{B_{a}(0)} K_{\epsilon, R}(y) f(x-y) dy \Big| \le$$
$$\Big| \int_{|x| \ge a} K_{\epsilon, R}(y) f(x-y) dy \Big| + \Big| \int_{B_{a}(0)} K_{\epsilon, R}(y) f(x-y) dy \Big| \le$$
$$\int_{|x| \ge a} |K_{\epsilon, R}(y) f(x-y)| dy + \Big| \int_{B_{a}(0) \setminus B_\epsilon(0)} K(y) f(x-y) dy \Big| \le$$
$$|M| \Big(A \int_{\{|x| \ge a\} \cap B_R(0)} \dfrac{1}{|a|^n} + |\int_{B_{a}(0) \setminus B_\epsilon(0)} K(y) dy| \Big) \le |M| \Big(\mu\{B_R(0)\} \cdot \dfrac{A}{|a|^n} + 0 \Big) = C(n)A \cdot |M|$$
for every $\epsilon > 0$, and for every $R \ge a$. Thus the integral is absolutetely convergent and the limit exists. Now, since $C_c^\infty(\mathbb{R}^n)$ is dense in $L^p(\mathbb{R}^n)$, we can approximate $g \in L^p(\mathbb{R}^n)$ with a sequence $C_c^\infty(\mathbb{R}^n) \ni g_j \rightarrow g$, with convergence in $L^p$. Thus, by the dominated convergence theorem, we have
$$\|K_{\epsilon, R} * g_j\|_{L^p}^p = \int_{\mathbb{R}^n} |(K_{\epsilon, R} * g_j)(x)|^p = \int_{\mathbb{R}^n} \Big| \int K_{\epsilon, R}(y) g_j(x-y) dy \Big|^p \nearrow$$
$$\int_{\mathbb{R}^n} \Big| \int K_{\epsilon, R}(y) g(x-y) dy \Big|^p \nearrow \int_{\mathbb{R}^n} \Big| \int K(y) g(x-y) dy \Big|^p =$$
$$\int_{\mathbb{R}^n} |(K * g)(x)|^p = \|K * g\|_{L^p}^p$$
as $\epsilon \rightarrow 0,\ R, j \rightarrow \infty$.
\end{proof}

\end{document}
