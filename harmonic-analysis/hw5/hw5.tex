\documentclass[12pt]{article}

\usepackage[margin=1in]{geometry}
\usepackage{amsmath, amsthm, amssymb, amsfonts, mathrsfs, unicode-math}

\newcommand{\N}{\mathbb{N}}
\newcommand{\Z}{\mathbb{Z}}

\newenvironment{theorem}[2][Theorem]{\begin{trivlist}
\item[\hskip \labelsep {\bfseries #1}\hskip \labelsep {\bfseries #2.}]}{\end{trivlist}}
\newenvironment{lemma}[2][Lemma]{\begin{trivlist}
\item[\hskip \labelsep {\bfseries #1}\hskip \labelsep {\bfseries #2.}]}{\end{trivlist}}
\newenvironment{exercise}[2][Exercise]{\begin{trivlist}
\item[\hskip \labelsep {\bfseries #1}\hskip \labelsep {\bfseries #2.}]}{\end{trivlist}}
\newenvironment{problem}[2][Problem]{\begin{trivlist}
\item[\hskip \labelsep {\bfseries #1}\hskip \labelsep {\bfseries #2.}]}{\end{trivlist}}
\newenvironment{question}[2][Question]{\begin{trivlist}
\item[\hskip \labelsep {\bfseries #1}\hskip \labelsep {\bfseries #2.}]}{\end{trivlist}}
\newenvironment{corollary}[2][Corollary]{\begin{trivlist}
\item[\hskip \labelsep {\bfseries #1}\hskip \labelsep {\bfseries #2.}]}{\end{trivlist}}

% -----------------------------------------
% #########################################
% -----------------------------------------
% INTERMISSION
% -----------------------------------------
% #########################################
% -----------------------------------------

\begin{document}

\title{Harmonic Analysis}
\title{Homework #5}
\author{Dan Sokolsky}

\maketitle

\begin{exercise}{1}
\end{exercise}

\begin{proof}
  First, we claim $f_b \in L^1(\mathbb{R}^n)$, and $f_s \in L^r(\mathbb{R}^n)$ -
  $$\int |f_b| = \int |f_b|^p |f_b|^{1-p} \le C(p) \|f\|^p_p < \infty$$
  $$\int |f_s|^r = \int |f_s|^{r-p} |f_s|^p \le C(r, p) \|f\|^p_p < \infty$$
  Now, since $|Tf| \le |Tf_b| + |Tf_s|$, we have
  $\{|Tf| > t\} \subseteq \{|Tf_b| > \frac{t}{2}\} \cup \{|Tf_s| > \frac{t}{2}\}$, so that -

  $$\mu\{|Tf| > t\} \le \mu\{|Tf_b| > \frac{t}{2}\} + \mu\{|Tf_s| > \frac{t}{2}\} \le$$
  $$\dfrac{2A \|f\|_1}{t} \int |f_b| + \dfrac{2^rA^r \|f\|_r^r}{t^r} \int |f_s|^r$$
  Now,
  $$\int_0^\infty t^{q-1}t^{-1} \int_{|f| > t} |f| = \int_{\mathbb{R}^n} |f| \int_0^{|f|} t^{q-2} = \dfrac{1}{q-1} \int_{\mathbb{R}^n} |f||f|^{q-1} = \dfrac{\|f\|^q_q}{q-1}$$
  since $q > p > 1$, and
  $$\int_0^\infty t^{q-1}t^{-r} \int_{|f| \le t} |f|^r = \int_{\mathbb{R}^n} |f|^r \int_{|f|}^{\infty} t^{q-1-r} = \dfrac{1}{r-q} \int_{\mathbb{R}^n} |f|^r|f|^{q-r} = \dfrac{\|f\|^q_q}{r-q}$$
  since $q < r$. Altogether,
  $$\|Tf\|_q \le C \|f\|_q$$
\end{proof}

\begin{exercise}{2}
\end{exercise}

\begin{proof}
  $(a)$
  $$\int_{\mathbb{R}^n \setminus B_{2r}(0)} |K(x) - K(x-z)| dx = \int_{\mathbb{R}^n \setminus B_{2r}(0)} \Big| \int_{x-z}^{x} DK(t) dt \Big| dx \le$$
  $$\int_{\mathbb{R}^n \setminus B_{2r}(0)} \int_{x-z}^{x} \Big|DK(t)\Big| dt\ dx \le \int_{\mathbb{R}^n \setminus B_{2r}(0)} B |x|^{-n-1} |x - (x-z)| dx =$$
  $$B |z| \int_{\mathbb{R}^n \setminus B_{2r}(0)} |x|^{-n-1}dx = C(n)B$$\\
  $(b)$
  $$\int_{\mathbb{R}^n} \Big( K(x) - K(x - x_{\xi}) \Big) e^{-2 \pi i x \xi}\ dx = \int_{\mathbb{R}^n} K(x) e^{-2 \pi i x \xi}\ dx - \int_{\mathbb{R}^n} K(x-x_\xi) e^{-2 \pi i x \xi}\ dx =$$
  $$\hat{K}(\xi) - \int_{\mathbb{R}^n} K(x) e^{-2 \pi i (x + x_\xi) \xi}\ dx = \hat{K}(\xi) - e^{-i \pi} \int_{\mathbb{R}^n} K(x) e^{-2 \pi i x \xi}\ dx =$$
  $$\hat{K}(\xi) + \hat{K}(\xi) = 2 \hat{K}(\xi)$$\\
  $(c)$ Since $K$ vanishes outside the annulus $B_R(0) \setminus B_\epsilon(0)$, we have
  $$\Big| \int_{B_\frac{1}{|\xi|}(0)} K(x)e^{-2 \pi i x \xi} \Big| dx \le \int_{B_\frac{1}{|\xi|}(0) \setminus  (B_R(0) \cup B_\epsilon(0))} |K(x)e^{-2 \pi i x \xi}| dx \le$$
  $$A \int_{B_\frac{1}{|\xi|}(0) \setminus  (B_R(0) \cup B_\epsilon(0))} \dfrac{1}{|x|^n} = C(n)A$$\\
  $(d)$
  $$\Big| \int_{B_\frac{1}{|\xi|}(0)} K(x-x_\xi)e^{-2 \pi i x \xi} \Big| dx \le \int_{B_{\frac{1}{|\xi|}+x_\xi}(x_\xi) \setminus  (B_{R+x_\xi}(x_\xi) \cup B_{\epsilon+x_\xi}(x_\xi))} |K(x)e^{-2 \pi i (x+x_\xi) \xi}| dx =$$ $$\int_{B_{\frac{1}{|\xi|}+x_\xi}(x_\xi) \setminus  (B_{R+x_\xi}(x_\xi) \cup B_{\epsilon+x_\xi}(x_\xi))} |e^{-i \pi}| \cdot |K(x)e^{-2 \pi i (x) \xi}| dx \le C(n) A$$\\
  $(e)$
  $$\Big| \hat{K}(\xi) \Big| = \Big| \int_{\mathbb{R}^n} K(x) e^{-2 \pi i x \xi}\ dx \Big| = \Big| \int_{B_R(0) \setminus B_\epsilon(0)} K(x) e^{-2 \pi i x \xi}\ dx \Big| \le$$
  $$\int_{B_R(0) \setminus B_\epsilon(0)} |K(x) e^{-2 \pi i x \xi}| dx \le A \int_{B_R(0) \setminus B_\epsilon(0)} \frac{1}{|x|^n}\ dx = C(n) A$$
\end{proof}

\begin{exercise}{3}
\end{exercise}

\begin{proof}
$(a)$ $(i)$ $|K \mathbf{1}_{B_r \setminus B_\epsilon}(x)| \le |K(x)| \le A|x|^{-n}$\\
$(ii)$
$$\int_{\mathbb{R}^n \setminus B_{2r}(0)} |K_{\epsilon, R}(x) - K_{\epsilon, R}(x-z)| dx \le \int_{\mathbb{R}^n \setminus B_{2r}(0)} |K_{\epsilon, R}(x)| + |K_{\epsilon, R}(x-z)| dx \le$$
$$\dfrac{2A}{|\epsilon|^n} \cdot \mu \{B_R(0)\} = C(n) A$$\\
$(iii)$\\
$(b)$
$$\|K_{\epsilon, R} * f\|_p^p = \int |K_{\epsilon, R}(x-y) f(y)|^p dy \le \dfrac{A}{|\epsilon|^{np}} \int_{B_R(0) \setminus B_\epsilon(0)} |f|^p \le C(n, p) \|f\|_p^p$$
$(c)$ Since $f$ is smooth and has compact support, by the Extreme Value Theorem, $f$ achieves a minimum and a maximum on it's domain. Let $m, M$ be the minimal, and mmaximal values of $f$, respectively. WLOG, suppose $R \ge 2a$. Then,
$$|(K_{\epsilon, R} * f)(x)| = \Big| \int K_{\epsilon, R}(y) f(x-y) dy \Big| = \Big| \int_{|x| \ge 2a} K_{\epsilon, R}(y) f(x-y) dy + \int_{B_{2a}(0)} K_{\epsilon, R}(y) f(x-y) dy \Big| \le$$
$$\Big| \int_{|x| \ge 2a} K_{\epsilon, R}(y) f(x-y) dy \Big| + \Big| \int_{B_{2a}(0)} K_{\epsilon, R}(y) f(x-y) dy \Big| \le$$
$$\int_{|x| \ge 2a} |K_{\epsilon, R}(y) f(x-y)| dy + \Big| \int_{B_{2a}(0) \setminus B_\epsilon(0)} K(y) f(x-y) dy \Big| \le$$
$$M \Big(A \int_{|x| \ge 2a} \dfrac{1}{2^n|a|^n} + |\int_{B_{2a}(0) \setminus B_\epsilon(0)} K(y) dy| \Big) = C(n)A \cdot M + 0 = C(n)A \cdot M$$
for every $\epsilon > 0$. Thus the integral is absolutetely convergent and the limit exists. Now, since $C_c^\infty(\mathbb{R}^n)$ is dense in $L^p(\mathbb{R}^n)$, we can approximate $g \in L^p(\mathbb{R}^n)$ with a sequence\\ $C_c^\infty(\mathbb{R}^n) \ni g_j \rightarrow g$, with convergence in $L^p$. Thus
$$K * g = \lim_{\epsilon \rightarrow 0,\ R,j \rightarrow \infty} K_{\epsilon, R} * g_j$$
since convolution is a continuouos operator.
\end{proof}

\end{document}
