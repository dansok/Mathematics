\documentclass[12pt]{article}

\usepackage[margin=1in]{geometry}
\usepackage{amsmath, amsthm, amssymb, amsfonts, mathrsfs, unicode-math}

\newcommand{\N}{\mathbb{N}}
\newcommand{\Z}{\mathbb{Z}}

\newenvironment{theorem}[2][Theorem]{\begin{trivlist}
\item[\hskip \labelsep {\bfseries #1}\hskip \labelsep {\bfseries #2.}]}{\end{trivlist}}
\newenvironment{lemma}[2][Lemma]{\begin{trivlist}
\item[\hskip \labelsep {\bfseries #1}\hskip \labelsep {\bfseries #2.}]}{\end{trivlist}}
\newenvironment{exercise}[2][Exercise]{\begin{trivlist}
\item[\hskip \labelsep {\bfseries #1}\hskip \labelsep {\bfseries #2.}]}{\end{trivlist}}
\newenvironment{problem}[2][Problem]{\begin{trivlist}
\item[\hskip \labelsep {\bfseries #1}\hskip \labelsep {\bfseries #2.}]}{\end{trivlist}}
\newenvironment{question}[2][Question]{\begin{trivlist}
\item[\hskip \labelsep {\bfseries #1}\hskip \labelsep {\bfseries #2.}]}{\end{trivlist}}
\newenvironment{corollary}[2][Corollary]{\begin{trivlist}
\item[\hskip \labelsep {\bfseries #1}\hskip \labelsep {\bfseries #2.}]}{\end{trivlist}}

% -----------------------------------------
% #########################################
% -----------------------------------------
% INTERMISSION
% -----------------------------------------
% #########################################
% -----------------------------------------

\begin{document}

\title{Harmonic Analysis}
\title{Homework #2}
\author{Dan Sokolsky}

\maketitle

\begin{exercise}{1}
\end{exercise}

\begin{proof}
  Since $f \ne 0$, there exists a radius $r$, on which $\int_{B_r(0)} |f| = c > 0$. For $|x| > r$, we have $B_r(0) \subseteq B_{|x| + r}(x)$ and therefore
  $$Mf(x) \ge \dfrac{1}{\mu(B_{|x|+r}(x))} \int_{B_{|x|+r}(x)} |f| \ge \dfrac{c}{(|x|+r)^n}$$
  So that
  $$\int_{\mathbb{R}^n} |Mf(x)| \ge \int_{|x| > r} \dfrac{c}{(|x|+r)^n} = \infty$$
  Thus, $Mf \not\in L^1(\mathbb{R}^n)$.
\end{proof}

\begin{exercise}{2}
\end{exercise}

\begin{proof}
  $(a) (i)$
  $$0 = \|f\|_{L^{1, \infty}} = \sup_{\lambda > 0} \lambda \cdot \mu \{|f| > \lambda\} \iff \mu \{|f| > \lambda\} = 0 \text{ for all } \lambda > 0$$ $$\iff |f| = 0 \iff f = 0$$
  $(ii)$
  $$\|kf\|_{L^{1, \infty}} = \sup_{\lambda > 0} \lambda \cdot \mu \{|kf| > \lambda\} = |k| \cdot \sup_{\lambda > 0} \lambda \cdot \mu \{|f| > \lambda\} = |k| \cdot \|f\|_{L^{1, \infty}}$$
  $(iii)$ Since $|f| + |g| \ge |f + g|$, we have
  $$\{|f| > \dfrac{\lambda}{2}\} \cup \{|g| > \dfrac{\lambda}{2}\} \supseteq \{|f| + |g| > \lambda\} \supseteq \{|f+g| > \lambda\}$$
  so that
  $$2(\|f\|_{L^{1, \infty}} + \|g\|_{L^{1, \infty}}) = 2\|f\|_{L^{1, \infty}} + 2\|g\|_{L^{1, \infty}} =$$
  $$\sup_{\lambda > 0} \lambda \cdot \mu \{2|f| > \lambda\} + \sup_{\lambda > 0} \lambda \cdot \mu \{2|g| > \lambda\} = $$
  $$\sup_{\lambda > 0} \lambda \cdot \mu \{|f| > \dfrac{\lambda}{2}\} + \sup_{\lambda > 0} \lambda \cdot \mu \{|g| > \dfrac{\lambda}{2}\} \ge$$
  $$\sup_{\lambda > 0} \lambda \cdot \mu \{|f| + |g| > \lambda\} \ge \sup_{\lambda > 0} \lambda \cdot \mu \{|f + g| > \lambda\} =$$
  $$\|f + g\|_{L^{1, \infty}}$$
  $(b)$ For $x \in [0, 1]$, we  have
  $$|f_\ell(x)| = \dfrac{1}{\log \ell} |\sum_{j=1}^\ell \dfrac{1}{x \ell - j}| = \dfrac{1}{\log \ell} |\sum_{j=1}^\ell \dfrac{1}{j - x \ell}|$$
  $$= \dfrac{1}{\log \ell} |\sum_{j=1}^\ell \dfrac{1}{j} - \dfrac{x \ell}{j(x \ell - j)}| \ge \dfrac{1}{\log \ell}(\sum_{j=1}^\ell |\dfrac{1}{j}| - |\dfrac{x \ell}{j(x \ell - j)}|) =$$
  $$\dfrac{1}{\log \ell}(\sum_{j=1}^\ell |\dfrac{1}{j}| - \sum_{j=1}^\ell|\dfrac{x \ell}{j(x \ell - j)}|) \ge \dfrac{1}{\log \ell} \sum_{j=1}^\ell \dfrac{1}{j} \ge$$
  $$\dfrac{1}{\log \ell} \sum_{j=1}^{\ell - 1} \int_{j}^{j+1}\dfrac{1}{j} = \dfrac{1}{\log \ell} \cdot \log  \ell = 1$$
  So that $\|f_\ell \|_{L^{1, \infty}} \ge 1 \cdot \mu\{|f_\ell| > 1\} \ge \mu[0, 1] = 1$.\\
  $(c)$ Since $\mu$ is translation invariant, we have
  $$\mu \Bigg\{\Bigg | \dfrac{1}{x - \frac{j}{\ell}} \Bigg | > \lambda \Bigg\} = \mu \Bigg\{\Bigg |\dfrac{1}{x} \Bigg | > \lambda \Bigg\}$$
  So that
  $$\Bigg \|\dfrac{1}{x - \frac{j}{\ell}} \Bigg\|_{L^{1, \infty}} = \Bigg \|\dfrac{1}{x} \Bigg\|_{L^{1, \infty}} = \sup_{\lambda > 0} \lambda \cdot \mu \Big \{x \in \mathbb{R}: \Big| \frac{1}{x} \Big | > \lambda \Big \} =$$
  $$2 \sup_{\lambda > 0} \lambda \cdot \mu \Big \{x \in [0, \infty): \frac{1}{x} > \lambda \Big \} = 2 \sup_{\lambda > 0} \lambda \cdot \mu \Big \{x \in [0, \infty): x < \frac{1}{\lambda} \Big \} =$$
  $$2 \sup_{\lambda > 0} \lambda \cdot \mu[0, \frac{1}{\lambda}) = 2 \sup_{\lambda > 0} \lambda \cdot \frac{1}{\lambda} = 2 \sup_{\lambda > 0} 1  = 2  \cdot 1 = 2$$
  So that
  $$\|f_\ell\|_{L^{1, \infty}} ֿ\le 2 \sum_{j=1}^\ell = \frac{1}{\ell \cdot \log \ell} \|\frac{1}{x}\|_{L^{1, \infty}} = \frac{4 \ell}{\ell \cdot \log \ell} = \frac{4}{\log \ell}$$
  Now, if there exists a norm $\|\cdot\|'$ such that $c\|f_\ell\|' \le \|f_\ell\|_{L^{1, \infty}} \le C\|\f_\ell\|'$, such that $c,C > 0$, then
  $$c\|f_\ell\|' \le \|f_\ell\|_{L^{1, \infty}} \le \dfrac{4}{\log \ell} \rightarrow 0$$
  as $\ell \rightarrow \infty$. Since $c > 0$, it follows that $\|f_\ell\|' = 0$, which is a contradiction since $|f_\ell| \ge 1$ on $[0, 1]$ for all $\ell$. Meaning, $f \ne 0$, so  that $\|\cdot\|'$ isn't a proper  norm.
\end{proof}

\begin{exercise}{3}
\end{exercise}

\begin{proof}
  $(a)\ (i)$ Since $\{B_r(x) | x \in \mathbb{R}^n\} \subseteq \{B_r(y) | x  \in B_r(y)\}$, we have
  $$M_1f = \sup_{\{B_r(y) | x  \in B_r(y)\}} \dfrac{1}{\mu(B_r(y))} \int_{B_r(y)} |f| \ge \sup_{\{B_r(x) | x \in \mathbb{R}\}} \dfrac{1}{\mu(B_r(x))} \int_{B_r(x)} |f| = Mf$$
  Now, for every $z \in B_r(y)$, $|x - z| \le |x - y| + |y - z| \le r + r = 2r$ so that $B_r(y) \subseteq B_{2r}(x)$. Now,
  $$M_1f = \sup_{\{B_r(y) | x  \in B_r(y)\}} \dfrac{1}{\mu(B_r(y))} \int_{B_r(y)} |f| \le \sup_{\{B_r(x) | x  \in \mathbb{R}^n\}} \dfrac{1}{\mu(B_r(x))} \int_{B_{2r}(x)} |f| =$$ $$\sup_{\{B_r(x) | x  \in \mathbb{R}^n\}} \dfrac{2^n}{\mu(B_r(x))} \int_{B_{2r}(x)} |f| = 2^n Mf$$
  So that $Mf \le M_1f \le 2^n Mf$.\\
  $(ii)$ $B_r(x) \subseteq Q_r(x) \subseteq B_{2r}(x)$. Let $c_2 = \frac{\mu(B_r(x))}{\mu(Q_r(x))}$. Observe that $c_2$ is independent of $r$. Then,
  $$M_2f = \sup_{\{Q_r(x) | x  \in \mathbb{R}^n\}} \dfrac{1}{\mu(Q_r(x))} \int_{Q_r(x)} |f| \ge \sup_{\{B_r(x) | x \in \mathbb{R}\}} \dfrac{c_2}{\mu(B_r(x))} \int_{B_r(x)} |f| = c_2Mf$$
  and
  $$M_2f = \sup_{\{Q_r(x) | x  \in \mathbb{R}^n\}} \dfrac{1}{\mu(Q_r(x))} \int_{Q_r(x)} |f| \le \sup_{\{B_{2r}(x) | x  \in \mathbb{R}^n\}} \dfrac{2^n}{\mu(B_{2r}(x))} \int_{B_{2r}(x)} |f| = 2^n Mf$$
  So that $c_2Mf \le M_2f \le 2^n Mf$.\\
  $(iii)$
\end{proof}

\end{document}
