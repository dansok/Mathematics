\documentclass[12pt]{article}

\usepackage[margin=1in]{geometry}
\usepackage{amsmath, amsthm, amssymb, amsfonts, mathrsfs, unicode-math}

\newcommand{\N}{\mathbb{N}}
\newcommand{\Z}{\mathbb{Z}}

\newenvironment{theorem}[2][Theorem]{\begin{trivlist}
\item[\hskip \labelsep {\bfseries #1}\hskip \labelsep {\bfseries #2.}]}{\end{trivlist}}
\newenvironment{lemma}[2][Lemma]{\begin{trivlist}
\item[\hskip \labelsep {\bfseries #1}\hskip \labelsep {\bfseries #2.}]}{\end{trivlist}}
\newenvironment{exercise}[2][Exercise]{\begin{trivlist}
\item[\hskip \labelsep {\bfseries #1}\hskip \labelsep {\bfseries #2.}]}{\end{trivlist}}
\newenvironment{problem}[2][Problem]{\begin{trivlist}
\item[\hskip \labelsep {\bfseries #1}\hskip \labelsep {\bfseries #2.}]}{\end{trivlist}}
\newenvironment{question}[2][Question]{\begin{trivlist}
\item[\hskip \labelsep {\bfseries #1}\hskip \labelsep {\bfseries #2.}]}{\end{trivlist}}
\newenvironment{corollary}[2][Corollary]{\begin{trivlist}
\item[\hskip \labelsep {\bfseries #1}\hskip \labelsep {\bfseries #2.}]}{\end{trivlist}}

% -----------------------------------------
% #########################################
% -----------------------------------------
% INTERMISSION
% -----------------------------------------
% #########################################
% -----------------------------------------

\begin{document}

\title{Harmonic Analysis}
\title{Homework #2}
\author{Dan Sokolsky}

\maketitle

\begin{exercise}{1}
\end{exercise}

\begin{proof}
  Since $f \ne 0$, there exists a radius $r$, on which $\int_{B_r(0)} |f| = c > 0$. For $|x| > r$, we have $B_r(0) \subseteq B_{|x| + r}(x)$ and therefore
  $$Mf(x) \ge \dfrac{1}{\mu(B_{|x|+r}(x))} \int_{B_{|x|+r}(x)} |f| \ge \dfrac{c}{(|x|+r)^n}$$
  So that
  $$\int_{\mathbb{R}^n} |Mf(x)| \ge \int_{|x| > r} \dfrac{c}{(|x|+r)^n} = \infty$$
  Thus, $Mf \not\in L^1(\mathbb{R}^n)$.
\end{proof}

\begin{exercise}{2}
\end{exercise}

\begin{proof}
  $(a) (i)$
  $$0 = \|f\|_{L^{1, \infty}} = \sup_{\lambda > 0} \lambda \cdot \mu \{|f| > \lambda\} \iff \mu \{|f| > \lambda\} = 0 \text{ for all } \lambda > 0$$ $$\iff |f| = 0 \iff f = 0$$
  $(ii)$
  $$\|kf\|_{L^{1, \infty}} = \sup_{\lambda > 0} \lambda \cdot \mu \{|kf| > \lambda\} = |k| \cdot \sup_{\lambda > 0} \lambda \cdot \mu \{|f| > \lambda\} = |k| \cdot \|f\|_{L^{1, \infty}}$$
  $(iii)$ Since $|f| + |g| \ge |f + g|$, we have
  $$\{|f| > \dfrac{\lambda}{2}\} \cup \{|g| > \dfrac{\lambda}{2}\} \supseteq \{|f| + |g| > \lambda\} \supseteq \{|f+g| > \lambda\}$$
  so that
  $$2(\|f\|_{L^{1, \infty}} + \|g\|_{L^{1, \infty}}) = 2\|f\|_{L^{1, \infty}} + 2\|g\|_{L^{1, \infty}} =$$ $$\sup_{\lambda > 0} \lambda \cdot \mu \{|f| > \dfrac{\lambda}{2}\} + \sup_{\lambda > 0} \lambda \cdot \mu \{|f| > \dfrac{\lambda}{2}\} \ge$$
  $$\sup_{\lambda > 0} \lambda \cdot \mu \{|f| + |g| > \lambda\} \ge \sup_{\lambda > 0} \lambda \cdot \mu \{|f + g| > \lambda\} =$$
  $$\|f + g\|_{L^{1, \infty}}$$
  $(b)$
\end{proof}

\begin{exercise}{3}
\end{exercise}

\begin{proof}
\end{proof}

\end{document}
