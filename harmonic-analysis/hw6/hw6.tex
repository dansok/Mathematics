\documentclass[12pt]{article}

\usepackage[margin=1in]{geometry}
\usepackage{amsmath, amsthm, amssymb, amsfonts, mathrsfs, unicode-math}

\newcommand{\N}{\mathbb{N}}
\newcommand{\Z}{\mathbb{Z}}

\newenvironment{theorem}[2][Theorem]{\begin{trivlist}
\item[\hskip \labelsep {\bfseries #1}\hskip \labelsep {\bfseries #2.}]}{\end{trivlist}}
\newenvironment{lemma}[2][Lemma]{\begin{trivlist}
\item[\hskip \labelsep {\bfseries #1}\hskip \labelsep {\bfseries #2.}]}{\end{trivlist}}
\newenvironment{exercise}[2][Exercise]{\begin{trivlist}
\item[\hskip \labelsep {\bfseries #1}\hskip \labelsep {\bfseries #2.}]}{\end{trivlist}}
\newenvironment{problem}[2][Problem]{\begin{trivlist}
\item[\hskip \labelsep {\bfseries #1}\hskip \labelsep {\bfseries #2.}]}{\end{trivlist}}
\newenvironment{question}[2][Question]{\begin{trivlist}
\item[\hskip \labelsep {\bfseries #1}\hskip \labelsep {\bfseries #2.}]}{\end{trivlist}}
\newenvironment{corollary}[2][Corollary]{\begin{trivlist}
\item[\hskip \labelsep {\bfseries #1}\hskip \labelsep {\bfseries #2.}]}{\end{trivlist}}

% -----------------------------------------
% #########################################
% -----------------------------------------
% INTERMISSION
% -----------------------------------------
% #########################################
% -----------------------------------------

\begin{document}

\title{Harmonic Analysis}
\title{Homework #6}
\author{Dan Sokolsky}

\maketitle

\begin{exercise}{1}
\end{exercise}

\begin{proof}
  $(1)\ (a)\ (i)$
  $$
    \hat{F}(\xi) =
    \mathcal{F}(f(\xi) - \Delta f(\xi)) =
    \mathcal{F}(f(\xi) - \sum_{j=1}^n \partial_i^2 f(\xi)) =
    \mathcal{F}(f(\xi)) - \mathcal{F}(\sum_{j=1}^n \partial_i^2 f(\xi))) =
  $$
  $$\mathcal{F}(f(\xi)) - \sum_{j=1}^n (2 \pi i \xi_j)^2 (\mathcal{F}f)(\xi))) =
    \hat{f}(\xi)\Big( 1 + 4\pi^2 |\xi|^2 \Big) = C\hat{f}(\xi)
  $$
  So  that $f(\xi) = \frac{1}{C} F(\xi) = \frac{1}{1 + 4\pi^2 |\xi|^2} F(\xi)$.\\
  $(ii)$
  $$
    \mathcal{F}^{-1}(m(\xi) \mathcal{F}((1 - \Delta)f)(\xi)) =
    \mathcal{F}^{-1}(m(\xi) \hat{F}(\xi)) =
    \mathcal{F}^{-1}(m(\xi) \cdot C\hat{f}(\xi)) =
    \mathcal{F}^{-1}((2 \pi i)^2 \xi_i \xi_j \hat{f}(\xi))) =
  $$
  $$
    \mathcal{F}^{-1}(-4 \pi^2 \xi_i \xi_j \hat{f}(\xi))) =
    \mathcal{F}^{-1}\Big( \mathcal{F}\Big( \partial_i \partial_j f(\xi) \Big) \Big) =
    \partial_i \partial_j f(\xi)
  $$
  $(b)$ We have,
  $$
    \|x^\alpha H(x)\|_\infty \le
    \|\mathcal{F}(x^\alpha H(x))\|_1 =
    \|\frac{1}{(-2\pi i)^{|\alpha|}} \partial^\alpha \hat{H}(x)\|_1 =
  $$
  $$
    -\dfrac{(|\alpha|)! \cdot (8\pi^2)^{|\alpha|}}{(-2\pi i)^{|\alpha|}} \int \dfrac{|\xi|^\alpha}{|1 +4\pi^2 |\xi|^2|^{|\alpha|+1}} d \xi <
    \infty
  $$
  for $|\alpha| \ge n-1$. So that $|H(x)| \le \frac{C}{|x|^\alpha}$ which represents an $L^1$ function on $\mathbb{R}^n \setminus \{0\}$, hence everywhere on  $\mathbb{R}^n$. Now,
  $$
    |f(x)| =
    |(H*F)(x)| =
    |\int H(y) F(x-y) dy| \le
    \|F\|_\infty |\int H(y)| \le
    \|H\|_1 \cdot \|F\|_\infty
  $$
  So that $\|f\|_\infty \le \|H\|_1 \cdot \|F\|_\infty$.\\
  $(c)$\\
    Letting $m(\xi) = -\frac{4 \pi^2 \xi_i \xi_j}{1 + 4\pi^2 |\xi|^2}$ as before, we see $m$ is symmetric. I.e., $m(\xi) = m(-\xi)$. By $(a)$,
    $$
      \partial_i \partial_j f(\xi) =
      \mathcal{F}^{-1}(m(\xi) \mathcal{F}((1 - \Delta)f)(\xi)) =
      (\check{m} * (1 - \Delta)f)(\xi)
    $$
    Keeping with the convention $\check{m}^{\#} (\xi) = \check{m} (-\xi)$, recall that $\check{m}^{\#}$ has the same properties as $\check{m}$, and in particular is also a proper multiplier. Now,
    $$
      |\partial_i \partial_j f  \cdot (1 - \Delta) g| =
      |\check{m} * (1-\Delta)f \cdot (1 - \Delta) g| =
    $$
    $$
      \Big|\int \int \check{m}(x-y) ((1 - \Delta) f)(y)\ dy\ ((1 - \Delta) g)(x)\ dx \Big| =
    $$
    $$
      \Big|\int \int \check{m}^{\#}(y-x) ((1 - \Delta) g)(x)\ dx\ ((1 - \Delta) f)(y)\ dy \Big| =
    $$
    $$
      \Big| \int \mathcal{F}(m(-y) \cdot \mathcal{F}((1 - \Delta) g)(y))\ ((1 - \Delta) f)(y)\ dy \Big| =
    $$
    $$
      \Big| \int \mathcal{F}(m(y) \cdot \mathcal{F}((1 - \Delta) g)(y))\ ((1 - \Delta) f)(y)\ dy \Big| =
    $$
    $$
      \Big| \int \partial_i \partial_j g(1-\Delta)f \Big| \le
      \int |\partial_i \partial_j g| \cdot |(1-\Delta)f| \le
      \|(1-\Delta)f\|_{L^1} \cdot \|\partial_i \partial_j   g\|_{L^\infty} \le
    $$
    $$
      C \|(1 - \Delta)f\|_{L^1} \cdot \|\Delta g\|_{L^\infty}
    $$
  The second inequality is Hölder's inequality.\\
  $(d)$
    $$
      \|\partial_i \partial_j f\|_{L^1} =
      \sup_{G \in \mathcal{S}:\ \|G\|_{L^\infty} \le 1} \int \partial_i \partial_j f G =
      \sup_{G \in \mathcal{S}:\ \|G\|_{L^\infty} \le 1} \int \partial_i \partial_j f (1-\Delta)g \le
    $$
    $$
      \sup_{G \in \mathcal{S}:\ \|G\|_{L^\infty} \le 1} C \|(1-\Delta)f\|_{L^1} \|\Delta g\|_{L^\infty} \le
      \sup_{G \in \mathcal{S}:\ \|G\|_{L^\infty} \le 1} C \|(1-\Delta)f\|_{L^1}\ \cdot\ C_2\|G\|_{L^\infty} \le
      C \|(1-\Delta)f\|_{L^1}
    $$
    Now, by $(a)$, we have $g = \frac{1}{C} G$. So that  $\|g\|_{L^\infty} = \frac{1}{C} \|G\|_{L^\infty} < \infty$. It follows that $|\partial_i^2 g| < \infty$, (else, $\partial_i g$, and hence $g$, explodes at some point). So that $\|\Delta  g\|_{L^{\infty}} = \|\sum_{i=1}^n \partial_i^2 g\|_{L^{\infty}} < \infty$ and $\|\Delta g\|_{L^\infty} \le C_1 \|g\|_{L^\infty} \le C \|G\|_{L^{\infty}}$.\\
    $(e)$
    $$
      \|(1-\Delta)f\|_1 =
      \int |f -\Delta f| dx =
      \int |\frac{1}{\lambda} f(\lambda x) - \frac{1}{\lambda} \sum_{j=1}^n \partial_j^2 f(\lambda x)| dx =
    $$
    $$
      \int |\frac{1}{\lambda} f(\lambda x) - \frac{1}{\lambda} \cdot \lambda^2 \sum_{j=1}^n \partial_j^2 f(\lambda x)| dx =
      \int |\frac{1}{\lambda} f(\lambda x) - \lambda \sum_{j=1}^n \partial_j^2 f(\lambda x)| dx \ge
    $$
    $$
      \Big| \frac{1}{\lambda} \int |f(\lambda x)| dx - \int |\lambda \sum_{j=1}^n \partial_j^2 f(\lambda x)| dx\ \Big| =
    $$
    $$
      \Big| \frac{1}{\lambda} \int |f(\lambda x)| dx - \|\Delta f\|_1\ \Big| \rightarrow
      \|0 - \Delta f\|_1 =
      \|\Delta f\|_1
    $$
    Likewise,
    $$
      \int |\frac{1}{\lambda} f(\lambda x) - \lambda \sum_{j=1}^n \partial_j^2 f(\lambda x)|\ dx \le
      \int |\frac{1}{\lambda} f(\lambda x)| + |\lambda \sum_{j=1}^n \partial_j^2 f(\lambda x)|\ dx \le
    $$
    $$
      \int |\frac{1}{\lambda} f(\lambda x)| + \int |\lambda \sum_{j=1}^n \partial_j^2 f(\lambda x)|\ dx =
      \int |\frac{1}{\lambda} f(\lambda x)| + \|\Delta f\|_1 \rightarrow
      0 + \|\Delta f\|_1 =
      \|\Delta f\|_1
    $$
    as $\lambda \rightarrow \infty$.
    Now, by $(d)$,
    $$
      \|D^2 f\|_1 \le
      C \|(1 - \Delta)f\|_1 =
      C \|\Delta\|_1
    $$
\end{proof}

\begin{exercise}{2}
\end{exercise}

\begin{proof}
  $(a)$ By continuity of the differential operator,
  $$
    \lim_{R \rightarrow \infty} (D^k f_R(x)) =
    D^k(\lim_{R \rightarrow \infty} f_R(x)) =
    D^k(\phi(0) \cdot e^{2 \pi i x \xi_0}) =
    \phi(0)  D^k e^{2 \pi i x \xi_0} =
  $$
  $$
    \phi(0) \Big( \sum \partial_{j_1} \cdots \partial_{j_k}(e^{2 \pi i x \xi_0}) \Big) =
    \phi(0) \cdot e^{2 \pi i x \xi_0}  \cdot \sum (\xi_0)_{j_1} \cdots (\xi_0)_{j_k} \ne 0
  $$
  and
  $$
    \lim_{R \rightarrow \infty} L f_R(x) =
    L(\lim_{R \rightarrow \infty} f_R(x)) =
    L(\phi(0) \cdot e^{2 \pi i x \xi_0}) =
    \sum_{|\alpha| \le k}  c_\alpha \partial^\alpha (\phi(0) \cdot e^{2 \pi i x \xi_0}) =
  $$
  $$
    \phi(0) \sum_{|\alpha| \le k}  c_\alpha (2 \pi i x \xi_0)^\alpha e^{2 \pi i x \xi_0}  =
    \phi(0) \cdot e^{2 \pi i x \xi_0} \sum_{|\alpha| \le k}  c_\alpha (2 \pi i x \xi_0)^\alpha =
    0
  $$
  Suppose by contradiction that $\|D^k f\|_p \le C  \|Lf\|_p$ for some nontrivial $f \in C^\infty_c$, with $\text{support}(f) = K$. Then,
  $$
    \|D^k f_R(x)\|_p =
    |R| \cdot \|D^k f(x)\|_p \le
    |R| \cdot C \|Lf(x)\|_p =
    C \|Lf_R(x)\|_p
  $$
  for all $R$, so
  $$
    \lim_{R \rightarrow \infty} \|D^k f_R\|_p =
    \mu(K) \cdot \phi(0) \sum (\xi_0)_{j_1} \cdots (\xi_0)_{j_k} >
    0 =
    \lim_{R \rightarrow \infty} \|Lf_R\|_p
  $$
  is a contradiction.\\\\
  $(b)$\\
  For  $i \le k$,
  $$
    \partial_{j_1} \cdots \partial_{j_i} f =
    \mathcal{F}^{-1}\Big( \dfrac{\xi{j_1} \cdots \xi{j_i}}{\sum_{|\alpha| \le k} c_\alpha \xi^{|\alpha|}} \cdot \mathcal{F}(Lf) \Big)
  $$
  where $m(\xi) = \frac{\xi{j_1} \cdots \xi{j_i}}{\sum_{|\alpha| \le k}c_\alpha \xi^{\alpha}}$. First, observe that $m \in L^\infty$. Let $p(\xi) = \frac{1}{m(\xi)}$. Then,
  $$
    p(\xi) =
    \frac{\sum_{|\alpha| \le k}c_\alpha \xi^{\alpha}}{\xi{j_1} \cdots \xi{j_i}} =
    \sum_{\ell \le k} \sum_{|\alpha| = \ell} \frac{c_\alpha \xi^{\alpha}}{\xi{j_1} \cdots \xi{j_i}} =
    \sum_{\ell \le k} \sum_{|\alpha| = l} p_\ell(\xi)
  $$
  where $p_\ell(\xi) = \frac{c_\alpha \xi^{\alpha}}{\xi{j_1} \cdots \xi{j_i}}$.
  Now,
  $$
    p_\ell(\lambda \cdot \xi) =
    \frac{\lambda^\ell c_\alpha \xi^{\alpha}}{\lambda^i \cdot \xi{j_1} \cdots \xi{j_i}} =
    \lambda^{\ell - i} \cdot p_\ell(\xi)
  $$
  Therefore, take $\lambda = \dfrac{1}{|\xi|^{\frac{|\beta| + \ell -i}{|\beta|}}}$. We have,
  $$
    \partial^\beta p(\xi) =
    \partial^\beta(\lambda^{\ell - i} \cdot p_\ell(\lambda \xi)) =
    \lambda^{|\beta| + \ell -i} \cdot \partial^\beta p_\ell(\lambda \xi)  =
  $$
  $$
    |\xi|^{-|\beta|} \cdot \partial^\beta p \Bigg( \frac{\xi}{|\xi|^{\frac{|\beta| + \ell -i}{|\beta|}}} \Bigg) \le
    c_\beta |\xi|^{-|\beta|}
  $$
  since $\mathbb{S}^{n-1}$ is compact $\Big(\frac{\xi}{|\xi|^{\frac{|\beta| + \ell -i}{|\beta|}}}$ takes values on a sphere, albeit not of radius $1 \Big)$.\\
  This means
  $$
    |\partial^\alpha p(\xi)| =
    |\sum_{\ell \le k} \sum_{|\alpha| = l} \partial^\alpha p_\ell(\xi)| \le
    \sum_{\ell \le k} \sum_{|\alpha| = l} |\partial^\alpha p_\ell(\xi)| \le
    c_\alpha |\xi|^{-|\alpha|}
  $$
  Finally,
  $$
    \partial^\alpha p(\xi) =
    \partial^\alpha \Big( \frac{1}{m(\xi)} \Big) =
    -\dfrac{\partial^\alpha m(\xi)}{m(\xi)^2} \iff
  $$
  $$
    |\partial^\alpha m(\xi)| =
    |\partial^\alpha p(\xi)| \cdot |m(\xi)|^2 \le
    c_\alpha |\xi|^{-|\alpha|} \cdot C \le
    c_\alpha |\xi|^{-|\alpha|}
  $$
  since $m$ is bounded. Now, by Hörmander-Mikhlin, it follows that
  $$
    \|\partial_{j_1} \cdots \partial_{j_i} f\|_{L^p} \le
    C \|Lf\|_{L^p}
  $$
  for every $0 \le i \le k$, as required.
\end{proof}

\begin{exercise}{3}
\end{exercise}

\begin{proof}

\end{proof}

\end{document}
